\documentclass{article}
\usepackage[landscape]{geometry}
\usepackage{url}
\usepackage{multicol}
\usepackage{amsmath}
\usepackage{esint}
\usepackage{bigints}
\usepackage{amsfonts}
\usepackage{xcolor}
\usepackage{tikz}
\usetikzlibrary{calc}
\usetikzlibrary{decorations.pathmorphing}
\usepackage{amsmath,amssymb}

\usepackage{colortbl}
\usepackage{xcolor}
\usepackage{mathtools}
\usepackage{amsmath,amssymb}
\usepackage{enumitem}
\usepackage{xhfill}
\makeatletter

\newcommand*\bigcdot{\mathpalette\bigcdot@{.5}}
\newcommand*\bigcdot@[2]{\mathbin{\vcenter{\hbox{\scalebox{#2}{$\m@th#1\bullet$}}}}}
\makeatother

\title{Exam 3 - Conservation of Energy Reference Sheet}
\usepackage[brazilian]{babel}
\usepackage[utf8]{inputenc}
%cambios de unidades (pt, mm, cm,ex,em,bp,dd,pc,sp) https://tex.stackexchange.com/questions/8260/what-are-the-various-units-ex-em-in-pt-bp-dd-pc-expressed-in-mm
\advance\topmargin-.8in
\advance\textheight3in
\advance\textwidth 3in
\advance\oddsidemargin -1.5in
\advance\evensidemargin -1.5in
\parindent 2pt
\parskip5pt
\newcommand{\hr}{\centerline{\rule{3.5in}{1pt}}}
%\colorbox[HTML]{e4e4e4}{\makebox[\textwidth-2\fboxsep][l]{texto}
\newcommand{\nc}[2][]{%
\tikz \draw [draw=black, ultra thick, #1]
    ($(current page.center)-(0.5\linewidth,0)$) -- 
    ($(current page.center)+(0.5\linewidth,0)$)
    node [midway, fill=white] {#2};
}% tomado de https://tex.stackexchange.com/questions/179425/a-new-command-of-the-form-tex
\begin{document}

\begin{center}{\huge{\textbf{PHY2048 Exam 3 Reference Sheet}}}\\
\end{center}
\begin{multicols*}{3}

\tikzstyle{mybox} = [draw=black, fill=white, very thick,
    rectangle, rounded corners, inner sep= 10pt, inner ysep=10pt]
\tikzstyle{fancytitle} =[fill=black, text=white, font=\bfseries]

%--------------------------
\begin{tikzpicture}
\node [mybox] (box){%
    \begin{minipage}{0.3\textwidth}
\begin{itemize}
\addtolength{\itemsep}{-2pt}
\item[$g = $]  
  9.80 $\frac{m}{s^2}$
\item[$J= $] 
    $N\cdot m$ 
\item[$ Hp = $] 
    745.7 Watts
\item[$G = $] 
   $ 6.67 x 10^{-11} N \cdot \frac{m^2}{kg^2}$
\item[$eV = $] 
   $ 1.6 \times 10^{-19}$

%\item[$R = $]
%  Constante de gases ideales$ = 8.31 \frac{\mbox{J}}{\mbox{mol K}} = 0.0821 \frac{\mbox{l atm}}{\mbox{mol K}}$
%\item[ $c_{\mbox{w}} = $] 
%  Calor específico agua $ = 1 {\mbox{ cal}}/({\mbox{g K}})$]
%\item[ $1 {\mbox{ cal}} $] $= 4.186 {\mbox{ J}}$]
%\item[ $\sigma = $] 
%  Constante Stefan-Boltzmann $ = 5.67\times 10^{-8}\>\frac{\mbox{W}}{\mbox{m}^2 \mbox{K}^4}$
\end{itemize}
    \end{minipage}
};
%---------------------------------
\node[fancytitle, right=10pt] at (box.north west) {Constants};
\end{tikzpicture}

%---------------------------
\begin{tikzpicture}
\node [mybox] (box){%
    \begin{minipage}{0.3\textwidth}
    $W = \Vec{F} \vec{d} \cdot cos(\theta)$\\
    $\theta = cos^{-1}(\frac{A \cdot B}{\mid A\mid \cdot \mid B \mid})$ \\
    $\rho = mv$ \\
    $m = \frac{\rho}{v} $ \\
    $m = \frac{2KE}{v^{2}}$ \\
    $V=k \sqrt{\frac{2KE}{m}}$ \\
    $KE = \frac{\rho^{2}}{2m}$ \\
    $\rho = \sqrt{2mKE}$ \\
    $k = \frac{1}{2}mv^{2}$ \\
    \begin{itemize}
        \item[$\Delta PE = $] 
            $PE_{final} - PE_{initial}$ \\
            $PE = mgh \Rightarrow (mass)(g)(height)$ \\
        \item[$KE = $] 
            $\frac{1}{2}mv{^2} = kg\frac{m}{s^{2}}\cdot m \Rightarrow N \cdot m \Rightarrow J$
        \item[$E_{tot}=$]
            $PE + KE = mgh + \frac{1}{2}mv^{2} \Rightarrow constant$
        \item[$C_{energy}= $]
            $ \frac{1}{2}kx^{2}_{i} = mgh$
        \item[$W_{f} = $]
            $ \mu_{k}mg \cdot d = \frac{1}{2}mv^{2}$
        \item[$P_{Watts}= $]
            $\frac{\Delta E}{\Delta t} \Rightarrow \frac{W}{\Delta t}$
    \end{itemize}
    \end{minipage}

};
%---------------------------------
\node[fancytitle, right= 10pt] at (box.north west) {Equations};
\end{tikzpicture}

%---------------------------
\begin{tikzpicture}
\node [mybox] (box){%
    \begin{minipage}{0.3\textwidth}
    $PE_{s \, to \, f} = mgh_{s}$\\
    $PE_{s \, to \, r} = mgh \Rightarrow mg(h_{s} - h_{r})$ \\
    $ Ex.01 \Rightarrow  (xx Kg)(9.8)(1.12 m - 0.96m) $ \\
    \end{minipage}
};
%---------------------------------
\node[fancytitle, right=10pt] at (box.north west) {Chapter 7 PPT Problems};
\end{tikzpicture}

%---------------------------
\begin{tikzpicture}
\node [mybox] (box){%
    \begin{minipage}{0.3\textwidth}
    \nc{Sliding Box}\\
    $\vec{F}\vec{d}\cdot cos(\theta)$\\
    \nc{Teeth Pulling}\\
    $(part\,a) W = mg\vec{d}$ \\$ (part\,b) \Rightarrow F_{teeth} = mg$\\

    \end{minipage}
};
%---------------------------------
\node[fancytitle, right=10pt] at (box.north west) {Chapter 7 Quiz};
\end{tikzpicture}

%---------------------------
\begin{tikzpicture}
\node [mybox] (box){%
    \begin{minipage}{0.3\textwidth}
    \nc{Angle Between Two Vectors}\\
    $\theta = cos^{-1}(\frac{A \cdot B}{\mid A \mid \cdot \mid B \mid})$
    \nc{Find the Dot Product, Given Magnitudes}\\
    $\vec{A}\cdot\vec{B} = cos(\theta)(\mid A \mid \cdot \mid B \mid)$\\  
    \nc{Evaluate The Work Graph}
    $ For\,Triangular\,Areas \rightarrow \frac{1}{2}b(in meters)h(in N)$ \\
    \nc{Work done By a Force in the x - Direction}
    $W = \int_{x=low}^{x=upper} \vec{F} \,w.r.t\,dx\,or\,dy$\\
    $Ex. = \int_{x=0}^{x=4.95} 3x\hat{i}$ \\
    $ = \int_{0}^{4.95}\frac{3}{2}(4.95)^{2}$\\
    \nc{Stone in The Well}
    $(part\,a) = PE = mgh$, where h = 0 = at the edge  \\
    $(part\,b) = PE = mgh$, where h = negative = dropped below the reference \\
    \end{minipage}
};
%------------ campo titulo---------------------
\node[fancytitle, right=10pt] at (box.north west) {Chapter 7 Quiz Continued};
\end{tikzpicture}
%---------------------------------
%\bigskip
%---------------------------
\begin{tikzpicture}
\node [mybox] (box){%
    \begin{minipage}{0.3\textwidth}
    \nc{Block Of Mass \& Spring}\\
    Goal: Solve For Height, h $\Rightarrow \frac{1}{2}kx^{2}_{i} = mgh $\\
    \nc{Rollercoaster Problem}
    $(Part\,a)$\,$\\
    \,Solve\,for\,each\,PE_{A,B,C} = PE = mgh$ \\
    $\,$\\

    $(Part\,b)\,$\\
    $Ex = KE_{B} = PE_{A} - PE_{B}$ \\
    $ \Rightarrow KE_{B} = \frac{1}{2}mv^{2}_{B}, solve for V_{B}$\\
    $\,$\\

    $(Part\,c)$ \\
    $W_{gravity\,A\,to\,C} = PE_{A} - PE_{C}$\\
    $\,$\\

    \nc{Sled's \& Crates}\\
    $(Part\,a)\,$\\
    $Ex = (\mu_{k})(m)(g)(d) = \frac{1}{2}mv^{2}$ \\
    $ Solve\,for\,d,\,where \Rightarrow (\mu_{k})(g)(d)=\frac{1}{2}v^{2}$\\
    $\,$\\

    $(Part\,b)\,$\\
    $Scalar Factor = \frac{New\,Velocity}{Old\,Velocity}$
    $\,$\\
    $d = (part\,a)(scalar\,factor)$\\
    $\,$\\

    $ \Rightarrow KE_{B} = \frac{1}{2}mv^{2}_{B}, solve for V_{B}$\\
    $\,$\\

    \end{minipage}
};
%------------ potencial titulo  ---------------------
\node[fancytitle, right=10pt] at (box.north west) {Chapter 8 Quiz};
\end{tikzpicture}

\newpage

%---------------------------
\begin{tikzpicture}
\node [mybox] (box){%
    \begin{minipage}{0.3\textwidth}
    \nc{Spring's \& Block's With Speed}
   $(Part\,a\, Frictionless)\,$\\
    $Solve\, for\, V_{f}, \,where \rightarrow kx^{2}_{i} = mv^{2}_{f}$
    $\,$\\
    $d = (part\,b\, COME BACK)(scalar\,factor)$\\
    $\,$\\

    \nc{Sliding Block With Falling Ball}
    $ Plug\,and\,Chug\,and \,Solve \,for \,v_{f}$\\
    $\,$\\
    $m_{2}gh= \frac{1}{2}(m_{1}+m_{2})(V^{2}_{f})+(\mu_{k}m_{1}g)(h)$\\
    $\,$\\

    \nc{Joules To Watts}
    $(1)\,J = \vec{F_{g}}\cdot\vec{d}$\\

    $(2)\,W = \frac{J}{\Delta t}$\\
    \nc{Power Of A Car Engine}

    $\frac{50mi}{h}\times\frac{1069.34m}{1mi}\times\frac{1h}{60 min}\times\frac{1\,min}{60\,sec}\times = \frac{m}{s}$\\
    $\Delta KE =\frac{1}{2}mv^{2}$\\
    $P_{Watts} = \frac{\Delta Ke}{t}$\\
    \nc{Older \& Newer Cars}
    $P_{Older} = \frac{E_{old}}{\Delta t} = \frac{\frac{1}{2} mv^{2}}{\Delta t} $\\
    $\frac{P_{new}}{P_{Old}} = (v^{2})$\\
    \end{minipage}
};
%---------------------------------
\node[fancytitle, right=10pt] at (box.north west) {Chapter 8 Quiz Continued};
\end{tikzpicture}

%\bigskip\\
%---------------------------
\begin{tikzpicture}
\node [mybox] (box){%
    \begin{minipage}{0.3\textwidth}
    \nc{Calculating Speed From Momentum}
    $\rho = \sqrt{2mKE}\, solve\,for\,m$\\
    $v = \sqrt(\frac{2KE}{m}),\, solve \,for \,v$\\
    \end{minipage}
};
%---------------------------------
\node[fancytitle, right=10pt] at (box.north west) {Chapter 9 Quiz};
\end{tikzpicture}


%\bigskip\\
%---------------------------
\begin{tikzpicture}
\node [mybox] (box){%
    \begin{minipage}{0.3\textwidth}
    \nc{Calculating Speed From Momentum}
    $\rho = \sqrt{2mKE}\, solve\,for\,m$\\
    $v = \sqrt(\frac{2KE}{m}),\, solve \,for \,v$\\
    \nc{Particle and Magnitude}
    $P_{x}=(m)(\hat{i})$\\
    $P_{y} = (m)(\hat{j})$\\
    $\rho = \sqrt{(\rho_{x})^{2}+(\rho_{y})^{2}}$\\
    $ \theta = tan^{-1}(\frac{\rho_{y}}{\rho_{x}})$\\
    \nc{Estimated Force-time Graph (Triangle)}
    $J =\frac{1}{2}(t_{b}-t_{a}\cdot \vec{F_{max}})$\\
    \nc{Average Force Over Interval (cont)}
    $\vec{F_{avg}}=\frac{J}{t_{b}-t_{a}} \Rightarrow \frac{F_{max}}{2}$\\

    \nc{Impulse and Energy Change in 1D - Ball Reversed}
    $J_{x}=\Delta\rho_{x} = mv_{f}=mv{i}$\\

    \nc{Complete Inelastic Collision in 2D - Stick}
    $\vec{v_f} = \frac{m_{1}v_{1i}+m_{2}v_{2i}}{m_{1}+m_{2}}$\\
    \nc{Inelastic Collision in Component Form}
    $\vec{v_fx} = \frac{m_{1}v_{1ix}+m_{2}v_{2ix}}{m_{1}+m_{2}}$\\
    
    $\vec{v_fy} = \frac{m_{1}v_{1iy}+m_{2}v_{2iy}}{m_{1}+m_{2}}$\\

    \nc{1D Elastic Collision of 2 Masses - Head On}
    $\vec{v_1f} = \frac{m_{1}-m_{2}}{m_{1}+m_{2}}v_{1i}+ \frac{2m_{2}}{m_{1}+m_{2}}v_{2i}$\\
    
    $\vec{v_1f} = \frac{2m{1}}{m_{1}+m_{2}}v_{1i}+ \frac{m_{2}-m_{1}}{m_{1}+m_{2}}v_{2i}$\\

    \end{minipage}
};
%---------------------------------
\node[fancytitle, right=10pt] at (box.north west) {Chapter 9 Quiz Continued};
\end{tikzpicture}
\bigskip
\bigskip


%---------------------------------
\begin{tikzpicture}
\node [mybox] (box){%
    \begin{minipage}{0.3\textwidth}
	\nc{Center of Mass Velocity - Fopr Any System}
    $\vec{v_compatibility} = \frac{m_{1}v_{1i}+m_{2}v_{2i}}{m_{1}+m_{2}}$\\
    $\vec{v_{cm,before}}v_{cm, before} = \vec{v_{cm, after}}$\\
	$B=\frac{\mu_oI}{4\pi R^2}\Delta l_{arco}$\\
	$B=\frac{\mu_oI}{4\pi R}(cos\theta_1-cos\theta_2)$\\
	\nc{Cargas en Movimiento}\\
	$r=\frac{mv}{qB}$\\
	$a=\frac{qvB}{m}=\frac{v^2}{r}$
	\end{minipage}
};
%---------------------------------
\node[fancytitle, right=10pt] at (box.north west) {Quiz Chapter 09 Continued};
\end{tikzpicture}
\bigskip




\end{multicols*}
\end{document}