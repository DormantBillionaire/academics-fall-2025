\documentclass[12pt]{amsart}

\addtolength{\hoffset}{-2.25cm}
\addtolength{\textwidth}{4.5cm}
\addtolength{\voffset}{-2.5cm}
\addtolength{\textheight}{5cm}
\setlength{\parskip}{0pt}
\setlength{\parindent}{0in}

\usepackage{amsthm,amsmath,amssymb}
\usepackage[colorlinks = true, linkcolor = blue, citecolor = red, final]{hyperref}
\usepackage{graphicx, multicol}
\usepackage{marvosym, wasysym}
\usepackage[figurename=Figure]{caption}
\usepackage{}
\usepackage{}
\usepackage{subfig}
\usepackage{physics}
\usepackage{enumitem}
\usepackage{siunitx}

\usepackage{tikz,bm}


\newcommand{\para}{\paragraph}
\newcommand{\ds}{\displaystyle}

\pagestyle{empty}
\begin{document}
\thispagestyle{empty}
{\scshape PHY2048} \hfill {\scshape \large Cumulative Exam Review} \hfill {Fall 2025\scshape}
 
\smallskip
\hrule

\bigskip
The following Exam Review is for my Physics with General Calculus I course for the Fall of 2025, which was taken during my gap year at Palm Beach State College under Professor Leo Bae. 

\bigskip

The content covered in the final exam will be a culmination of quiz problems from the following chapters: 
\begin{itemize}
    \item Unit 1: Chapters 1, 2, 3, and 4
    \item Unit 2: Chapters 5, 6, and 13
    \item Unit 3: Chapters 7, 8, and 9
\end{itemize}
\bigskip

\hrule
%%%%%%%%%%%%%%%%% CHAPTER 1 QUIZ SECTION %%%%%%%%%%%%%%%%%%

\paragraph{\large \bf Chapter 1 Quiz Problems } 

%%%%%%%%%%%% CHAPTER 1 PROBLEM 1 %%%%%%%%%%%%%%%%
\bigskip
\paragraph*{\bf Problem 1 -- Milky Way \& Andromeda Scale Model}
\smallskip

\paragraph{\bf Using Proportions for Scale Models}
\smallskip

The disk of the Milky Way galaxy is about $1.0 \times 10^{5}$ ly in diameter.  
The distance from the center of the Milky Way to the center of the Andromeda galaxy is about $2.0\times 10^{6}$ ly.

\begin{enumerate}[label=(\alph*)]
    \item \textbf{Distance between galaxy centers in the model}

    If the Milky Way is represented by a circular plate of diameter $d_{\text{MW,model}} = \SI{16}{cm}$, then the scale factor is
    \[
        \frac{d_{\text{MW,real}}}{d_{\text{MW,model}}}
        = \frac{1.0\times 10^{5}~\text{ly}}{16~\text{cm}}.
    \]
    For the model distance between the centers:
    \[
        \frac{d_{\text{MW,real}}}{d_{\text{And,real}}}
        = \frac{1.0\times 10^{5}~\text{ly}}{2.0\times 10^{6}~\text{ly}}
        = \frac{16~\text{cm}}{x_{\text{cm}}}.
    \]
    Solve for $x_{\text{cm}}$ and convert:
    \[
        x_{\text{m}} = \frac{x_{\text{cm}}}{100}.
    \]

    \item \textbf{Diameter of Local Group in the same scale}

    The Local Group has diameter $D_{\text{LG,real}} = 1.0\times 10^{7}~\text{ly}$.  
    Using the same scale (Milky Way $\to 16~\text{cm}$):
    \[
        \frac{d_{\text{MW,real}}}{D_{\text{LG,real}}}
        = \frac{1.0\times 10^{5}~\text{ly}}{1.0\times 10^{7}~\text{ly}}
        = \frac{16~\text{cm}}{D_{\text{LG,model (cm)}}}.
    \]
    Solve for $D_{\text{LG,model (cm)}}$, then convert:
    \[
        D_{\text{LG,model (m)}} = \frac{D_{\text{LG,model (cm)}}}{100}.
    \]
\end{enumerate}

\bigskip
%%%%%%%%%%%% CHAPTER 1 PROBLEM 4 %%%%%%%%%%%%%%%%
\hrule
\paragraph*{\bf Problem 4 -- Rods \& Cross Sections}
\medskip

\paragraph{\bf Mass of a Rod with Linearly Varying Density}
\smallskip

A thin rod extends from $x = 0$ to $x = \SI{16.0}{cm}$.  
Cross-sectional area $A = \SI{8.50}{cm^2}$.  
Density increases linearly from $\rho(0) = \SI{3.00}{g/cm^3}$ to $\rho(16~\text{cm}) = \SI{19.0}{g/cm^3}$.

Assume
\[
    \rho(x) = B + Cx,
\]
with $x$ in cm.

\begin{enumerate}[label=(\alph*)]
    \item \textbf{Find $B$ from $\rho(0)$}
    \[
        \rho(0) = B \quad\Rightarrow\quad B = \SI{3.00}{g/cm^3}.
    \]

    \item \textbf{Find $C$ from $\rho(16~\text{cm})$}
    \[
        \rho(16) = B + C(16) = \SI{19.0}{g/cm^3}.
    \]
    So
    \[
        19.0 = 3.00 + 16C \quad\Rightarrow\quad C = \frac{19.0 - 3.0}{16} = \SI{1.00}{g/cm^4}.
    \]

    \item \textbf{Total mass of the rod}

    Mass $m$ is
    \[
        m = \int_0^{16} \rho(x) A\,dx
          = A \int_0^{16} \left(B + Cx\right)\,dx.
    \]
    Compute:
    \begin{align*}
        m &= A \left[ Bx + \frac{C x^2}{2} \right]_0^{16} \\
          &= (\SI{8.50}{cm^2})\left[ (3.00)(16) + \frac{1.00(16)^2}{2} \right] \\
          &= 8.50\,(48 + 128)~\si{g} \\
          &= 8.50 \times 176~\si{g} \approx 1496~\si{g}.
    \end{align*}
    Convert to kilograms:
    \[
        m \approx 1.496~\si{kg}.
    \]
\end{enumerate}

%%%%%%%%%%%%%%%%% CHAPTER 2 QUIZ SECTION %%%%%%%%%%%%%%%%%%

\hrule
\bigskip
\paragraph{\large \bf Chapter 2 Quiz Problems } 
\bigskip

%%%%%%%%%%%% CHAPTER 2 PROBLEM 2 %%%%%%%%%%%%%%%%
\paragraph*{\bf Problem 2 -- Constant Speed Motion}
\smallskip

\paragraph{\bf Position and Velocity for Motion at Constant Speed}
\smallskip

For a particle moving at constant speed $v$ along a straight line:

\begin{enumerate}[label=(\alph*)]
    \item \textbf{1D motion (along $x$-axis)}
    \[
        x(t) = x_0 + vt,
    \]
    where $v$ is positive (right) or negative (left).  
    Average speed and instantaneous speed are both $\lvert v \rvert$.

    \item \textbf{2D motion at constant speed}

    If direction is given by angle $\theta$ from the $+x$ axis:
    \begin{align*}
        v_x &= v\cos\theta, \\
        v_y &= v\sin\theta, \\
        x(t) &= x_0 + v_x t, \\
        y(t) &= y_0 + v_y t.
    \end{align*}
\end{enumerate}

\bigskip
%%%%%%%%%%%% CHAPTER 2 PROBLEM 15 %%%%%%%%%%%%%%%%
\paragraph*{\bf Problem 15 -- Helicopters \& Heights}
\smallskip

\paragraph{\bf Vertical Motion: Position vs. Time}
\smallskip

A helicopter's height above the ground can be modeled in 1D (vertical $y$-axis):

\begin{enumerate}[label=(\alph*)]
    \item \textbf{Constant vertical speed}
    \[
        y(t) = y_0 + v_y t,
    \]
    where $v_y$ is positive when ascending and negative when descending.

    \textit{Example (constant speed):}  
    Suppose $y_0 = \SI{20}{m}$ (starting height) and the helicopter climbs straight up with
    $v_y = \SI{5}{m/s}$.  
    After $t = \SI{4}{s}$:
    \[
        y(4) = 20 + (5)(4) = \SI{40}{m}.
    \]

    To find the time to reach $y = \SI{80}{m}$:
    \[
        y = y_0 + v_y t \quad\Rightarrow\quad
        80 = 20 + 5t \Rightarrow t = \frac{60}{5} = \SI{12}{s}.
    \]

    \item \textbf{With constant vertical acceleration (e.g., gravity)}
    \[
        y(t) = y_0 + v_{0y} t + \frac{1}{2} a_y t^2,
    \]
    with
    \[
        v_y(t) = v_{0y} + a_y t.
    \]

    \textit{Example (constant acceleration):}  
    Let $y_0 = \SI{50}{m}$, $v_{0y} = \SI{2.0}{m/s}$ (upward), and $a_y = -\SI{9.8}{m/s^2}$ (gravity downward).
    \begin{align*}
        y(1.0) &= 50 + (2.0)(1.0) + \frac{1}{2}(-9.8)(1.0)^2 \\
               &= 50 + 2.0 - 4.9 = \SI{47.1}{m}, \\
        v_y(1.0) &= 2.0 + (-9.8)(1.0) = -\SI{7.8}{m/s}
        \quad\text{(moving downward after 1 s).}
    \end{align*}

    Use these forms to solve for \emph{time}, \emph{height}, or \emph{velocity} when the helicopter passes a certain point or reaches a specified altitude.
\end{enumerate}


%%%%%%%%%%%%%%%%% CHAPTER 3 QUIZ SECTION %%%%%%%%%%%%%%%%%%

\bigskip
\paragraph{\large \bf Chapter 3 Quiz Problems } 
\bigskip
\hrule
\bigskip

%%%%%%%%%%%% CHAPTER 3 PROBLEMS 1–3 %%%%%%%%%%%%%%%%
\paragraph*{\bf Problems 1, 2, \& 3 -- Polar $\leftrightarrow$ Cartesian Coordinates}
\smallskip

\paragraph{\bf Converting Between Polar and Cartesian in 2D}
\smallskip

\begin{enumerate}[label=(\alph*)]
    \item \textbf{Given polar coordinates $(r,\theta)$}

    For a point with radius $r$ and angle $\theta$ (measured from the $+x$ axis):
    \[
        x = r\cos\theta, \qquad y = r\sin\theta.
    \]

    \item \textbf{Given partial Cartesian information}

    If you know $x$ and $\theta$:
    \[
        x = r\cos\theta \quad\Rightarrow\quad r = \frac{x}{\cos\theta}.
    \]
    Then
    \[
        y = r\sin\theta = \frac{x}{\cos\theta}\sin\theta
          = x\tan\theta.
    \]

    \item \textbf{Going from Cartesian to polar}

    If $x$ and $y$ are known:
    \[
        r = \sqrt{x^2 + y^2}, \qquad \theta = \tan^{-1}\!\left(\frac{y}{x}\right)
    \]
    (adjust $\theta$ based on the quadrant).
\end{enumerate}

\bigskip
%%%%%%%%%%%% CHAPTER 3 PROBLEM 4 %%%%%%%%%%%%%%%%
\paragraph*{\bf Problem 4 -- Forces and Boxes}
\smallskip

\paragraph{\bf Vector Addition of Two Forces on a Box}
\smallskip

\begin{figure}[h!]
    \centering
    \includegraphics[width=0.35\linewidth]{/Users/aspenjohnson/Documents/GitHub/academics-fall-2025/PHY2048_GeneralPhysicsI/figures/Chapter3_ForcesAndBoxes.png}
    \caption{Forces $\vec{F}_1$ and $\vec{F}_2$ acting on a box at angle(s) $\theta_1$, $\theta_2$.}
    \label{fig:TwoStringTwoForces}
\end{figure}

Let $\vec{F}_1$ have magnitude $F_1$ and angle $\theta_1$ from the $+x$ axis.  
Let $\vec{F}_2$ be vertical (for example), or at angle $\theta_2$.

\begin{enumerate}[label=(\alph*)]
    \item \textbf{Resolve each force into components}
    \begin{align*}
        F_{1x} &= F_1 \cos\theta_1, &
        F_{1y} &= F_1 \sin\theta_1, \\
        F_{2x} &= F_2 \cos\theta_2, &
        F_{2y} &= F_2 \sin\theta_2.
    \end{align*}
    (If $\vec{F}_2$ is purely vertical, then $F_{2x} = 0$ and $F_{2y} = F_2$.)

    \item \textbf{Resultant force components}
    \begin{align*}
        F_{Rx} &= F_{1x} + F_{2x}, \\
        F_{Ry} &= F_{1y} + F_{2y}.
    \end{align*}

    \item \textbf{Magnitude and direction of resultant}
    \begin{align*}
        F_R &= \sqrt{F_{Rx}^2 + F_{Ry}^2}, \\
        \theta_R &= \tan^{-1}\!\left(\frac{F_{Ry}}{F_{Rx}}\right)
    \end{align*}
    (again, adjust $\theta_R$ based on the quadrant of $\vec{F}_R$).
\end{enumerate}

%%%%%%%%%%%%%%%%% CHAPTER 5 QUIZ SECTION %%%%%%%%%%%%%%%%%%

\bigskip
\paragraph{\large \bf Chapter 5 Quiz Problems } 
\bigskip
\hrule
\bigskip

%%%%%%%%%%%% CHAPTER 5 PROBLEM 1 %%%%%%%%%%%%%%%%
\paragraph*{\bf Problem 1 -- Resultant Forces \& Magnitudes}
\smallskip

\paragraph{\bf From Cartesian Components to Magnitude (and Direction)}
\smallskip

Given a mass $m$ and acceleration
\[
\vec{a} = a_x \hat{i} + a_y \hat{j},
\]
the net force is
\[
\vec{F} = m \vec{a} = (m a_x)\hat{i} + (m a_y)\hat{j}.
\]

\begin{enumerate}[label=(\alph*)]
    \item \textbf{Find components of $\vec{F}$}
    \[
        F_x = m a_x, \qquad F_y = m a_y.
    \]

    \item \textbf{Magnitude of the force}
    \[
        \lvert \vec{F} \rvert = \sqrt{F_x^2 + F_y^2}.
    \]

    \item (Optional) \textbf{Direction (angle from $+x$ axis)}
    \[
        \theta = \tan^{-1}\!\left(\frac{F_y}{F_x}\right).
    \]
\end{enumerate}

\bigskip
%%%%%%%%%%%% CHAPTER 5 PROBLEM 2 %%%%%%%%%%%%%%%%
\paragraph*{\bf Problem 2 -- Normal Force on a Box with Hanging Weights}
\smallskip

\paragraph{\bf Vector Addition of Two Vertical Forces on a Box}
\smallskip

Let the box weight be $W_{\text{box}}$ and the upward tension from the hanging mass be $T$. Taking up as positive, the normal force $F_N$ satisfies:
\[
F_N + T - W_{\text{box}} = 0 \quad\Rightarrow\quad F_N = W_{\text{box}} - T.
\]

\begin{enumerate}[label=(\alph*)]
    \item \textbf{Case A: Box on the ground, no cord attached}
    \[
        F_N = W_{\text{box}} = \SI{15.5}{lb}
    \]
    (normal force equals the weight).

    \item \textbf{Case B: Hanging weight smaller than box weight}
    \[
        F_N = W_{\text{box}} - W_{\text{hang}}.
    \]
    The cord \emph{reduces} the load on the floor but cannot lift the box.

    \item \textbf{Case C: Hanging weight larger than box weight}
    \[
        W_{\text{hang}} > W_{\text{box}} \quad\Rightarrow\quad F_N = 0
    \]
    (the box is lifted off the floor, so the floor exerts no force).
\end{enumerate}

\bigskip
%%%%%%%%%%%% CHAPTER 5 PROBLEM 3 %%%%%%%%%%%%%%%%
\paragraph*{\bf Problem 3 -- Sailboats \& Constant Velocity}
\smallskip

\paragraph{\bf Constant Velocity $\Rightarrow$ Net Force is Zero}
\smallskip

\begin{figure}[h!]
    \centering
    \includegraphics[width=0.35\linewidth]{/Users/aspenjohnson/Documents/GitHub/academics-fall-2025/PHY2048_GeneralPhysicsI/figures/Chapter5_SailboatsAndForces.png}
    \caption{Sailboat being pulled by forces $\vec{F}$, $\vec{n}$, and $\vec{P}$.}
    \label{fig:SailboatForces}
\end{figure}

For constant velocity,
\[
\sum \vec{F} = \vec{0}.
\]

Given a drag force
\[
\vec{F} = F_x \hat{i} + F_y \hat{j},
\]
and two other forces $\vec{n}$ and $\vec{P}$, we require
\[
\vec{F} + \vec{n} + \vec{P} = \vec{0}.
\]

\begin{enumerate}[label=(\alph*)]
    \item \textbf{Resolve the known force $\vec{F}$ into components}
    \[
        F_x = F\cos\theta, \qquad F_y = F\sin\theta
    \]
    (signs depend on the direction of $\vec{F}$).

    \item \textbf{Write component equations}
    \[
        F_x + n_x + P_x = 0, \qquad F_y + n_y + P_y = 0,
    \]
    and solve for the magnitudes of $\vec{n}$ and $\vec{P}$.
\end{enumerate}

\bigskip
%%%%%%%%%%%% CHAPTER 5 PROBLEM 4 %%%%%%%%%%%%%%%%
\paragraph*{\bf Problem 4 -- Stabilizing Broken Legs (Traction System)}
\smallskip

\paragraph{\bf Leg Traction System Tension}
\smallskip

\begin{figure}[h!]
    \centering
    \includegraphics[width=0.35\linewidth]{/Users/aspenjohnson/Documents/GitHub/academics-fall-2025/PHY2048_GeneralPhysicsI/figures/Chapter5_StabilizingABrokenLeg.png}
    \caption{Leg traction system with pulleys and a hanging mass $m$.}
    \label{fig:LegTraction}
\end{figure}

\begin{enumerate}[label=(\alph*)]
    \item \textbf{Tension from hanging mass}
    \[
        T = mg
    \]
    because the mass is in equilibrium and its weight is supported by the rope.

    \item \textbf{Horizontal force pulling the leg}
    If two segments of rope pull on the leg at angle $\theta$ above/below the horizontal:
    \[
        F_{\text{right}} = 2T\cos\theta.
    \]
\end{enumerate}

\bigskip
%%%%%%%%%%%% CHAPTER 5 PROBLEM 5 %%%%%%%%%%%%%%%%
\paragraph*{\bf Problem 5 -- Chin-up Forces from a Velocity--Time Graph}
\smallskip

\paragraph{\bf Convert Graph Slope to Acceleration, then Use $F = m(g + a_y)$}
\smallskip

\begin{figure}[h!]
    \centering
    \includegraphics[width=0.35\linewidth]{/Users/aspenjohnson/Documents/GitHub/academics-fall-2025/PHY2048_GeneralPhysicsI/figures/Chapter5_SpeedOverTimeCurve.png}
    \caption{Speed--time curve for the chin-up motion. Convert \si{cm/s} to \si{m/s}.}
    \label{fig:SpeedOverTime}
\end{figure}

\begin{enumerate}[label=(\alph*)]
    \item \textbf{Find acceleration from the slope}
    \[
        a_y = \frac{\Delta v_y}{\Delta t}
    \]
    using the velocity--time graph (convert \si{cm/s} to \si{m/s}).

    \item \textbf{Force from the bar}
    \[
        F_{\text{bar}} = N = mg + ma_y = m(g + a_y).
    \]
    For $a_y = 0$ (flat part of the graph), $F_{\text{bar}} = mg$ (true bodyweight).
\end{enumerate}

\bigskip
%%%%%%%%%%%% CHAPTER 5 PROBLEM 6 %%%%%%%%%%%%%%%%
\paragraph*{\bf Problem 6 -- Tension in Elevator Strings}
\smallskip

\paragraph{\bf Two Masses Suspended in an Accelerating Elevator}
\smallskip

\begin{figure}[h!]
    \centering
    \includegraphics[width=0.30\linewidth]{/Users/aspenjohnson/Documents/GitHub/academics-fall-2025/PHY2048_GeneralPhysicsI/figures/Chapter5_ElevatorsAndTension.png}
    \caption{Two masses $m$ suspended in an elevator, held by tensions $T_1$ and $T_2$.}
    \label{fig:elevatortension}
\end{figure}

\begin{enumerate}[label=(\alph*)]
    \item \textbf{Lower string (tension $T_2$) supports one mass}
    \[
        T_2 - mg = ma \quad\Rightarrow\quad T_2 = m(g + a).
    \]

    \item \textbf{Upper string (tension $T_1$) supports two masses}
    \[
        T_1 - 2mg = 2ma \quad\Rightarrow\quad T_1 = 2m(g + a).
    \]

    \item \textbf{Max elevator acceleration} comes from setting $T_1$ or $T_2$ equal to the maximum safe tension and solving for $a$.
\end{enumerate}

%%%%%%%%%%%%%%%%% CHAPTER 6 QUIZ SECTION %%%%%%%%%%%%%%%%%%

\newpage
\paragraph{\large \bf Chapter 6 Quiz Problems } 
\bigskip
\hrule
\bigskip

%%%%%%%%%%%% CHAPTER 6 PROBLEM 1 %%%%%%%%%%%%%%%%
\paragraph*{\bf Problem 1 -- Station Rotation}
\smallskip

\paragraph{\bf Artificial Gravity in a Rotating Space Station}
\smallskip

\begin{enumerate}[label=(\alph*)]
    \item Given diameter $D = \SI{119}{m}$, radius $r = D/2 = \SI{59.5}{m}$ and desired centripetal acceleration $a_c = \SI{3.60}{m/s^2}$:
    \begin{align*}
        a_c &= \omega^2 r \\
        \omega &= \sqrt{\frac{a_c}{r}} \\
        f &= \frac{\omega}{2\pi} \quad\Rightarrow\quad
        \text{rev/min} = 60 f \approx 2.35~\text{rev/min}.
    \end{align*}

    \item For $a_c = g = \SI{9.80}{m/s^2}$ with the same radius:
    \begin{align*}
        \omega &= \sqrt{\frac{g}{r}}, \quad
        f = \frac{\omega}{2\pi}, \quad
        \text{rev/min} = 60 f \approx 3.88~\text{rev/min}.
    \end{align*}
\end{enumerate}

\bigskip
%%%%%%%%%%%% CHAPTER 6 PROBLEM 2 %%%%%%%%%%%%%%%%
\paragraph*{\bf Problem 2 -- Coins on a Turntable}
\smallskip

\paragraph{\bf Static Friction as Centripetal Force}
\smallskip

\begin{enumerate}[label=(\alph*)]
    \item The force providing centripetal acceleration is \textbf{static friction}:
    \[
        F_c = f_s = \mu_s N.
    \]

    \item Given radius $r = \SI{0.294}{m}$ and speed $v = \SI{0.484}{m/s}$:
    \begin{align*}
        a_c &= \frac{v^2}{r}, \\
        \mu_s &= \frac{a_c}{g} = \frac{v^2}{rg} \approx 0.0813.
    \end{align*}
\end{enumerate}

\bigskip
%%%%%%%%%%%% CHAPTER 6 PROBLEM 3 %%%%%%%%%%%%%%%%
\paragraph*{\bf Problem 3 -- Apollo Astronaut in Lunar Orbit}
\smallskip

\paragraph{\bf Circular Orbit Using $a_c = g_{\text{orbit}}$}
\smallskip

\begin{enumerate}[label=(\alph*)]
    \item Radius of orbit:
    \begin{align*}
        R_{\text{Moon}} &= \SI{1.70e6}{m}, \quad h = \SI{4.43e5}{m}, \\
        r &= R_{\text{Moon}} + h.
    \end{align*}
    Given $a_c = g_{\text{orbit}} = \SI{1.07}{m/s^2}$ and $a_c = v^2/r$:
    \[
        v = \sqrt{a_c r} \approx \SI{1.51e3}{m/s}.
    \]

    \item Orbital period:
    \[
        T = \frac{2\pi r}{v} \approx \SI{8.89e3}{s}.
    \]
\end{enumerate}

\bigskip
%%%%%%%%%%%% CHAPTER 6 PROBLEM 4 %%%%%%%%%%%%%%%%
\paragraph*{\bf Problem 4 -- Space Station Rotation (Again)}
\smallskip

\paragraph{\bf Artificial Gravity with Different Diameter}
\smallskip

\begin{enumerate}[label=(\alph*)]
    \item Diameter $D = \SI{138}{m}$, radius $r = \SI{69.0}{m}$, $a_c = \SI{4.00}{m/s^2}$:
    \begin{align*}
        a_c &= \omega^2 r \Rightarrow \omega = \sqrt{\frac{a_c}{r}} \\
        f &= \frac{\omega}{2\pi}, \quad
        \text{rev/min} = 60f \approx 2.30~\text{rev/min}.
    \end{align*}
\end{enumerate}

\bigskip
%%%%%%%%%%%% CHAPTER 6 PROBLEM 5 %%%%%%%%%%%%%%%%
\paragraph*{\bf Problem 5 -- Stuntman on a Rope Swing}
\smallskip

\paragraph{\bf Tension at Bottom of Circular Arc}
\smallskip

\begin{enumerate}[label=(\alph*)]
    \item Given $m = \SI{86.5}{kg}$, $L = \SI{12.0}{m}$, $v = \SI{8.60}{m/s}$, $T_{\max} = \SI{1000}{N}$.
    At the bottom of the swing:
    \[
        \sum F_r = T - mg = \frac{mv^2}{L} \quad\Rightarrow\quad
        T = mg + \frac{mv^2}{L}.
    \]
    Numerically, $T \approx \SI{1.38e3}{N} > \SI{1000}{N}$, so the rope \textbf{breaks} at $v = \SI{8.60}{m/s}$.

    \item Maximum safe speed (set $T = T_{\max}$):
    \begin{align*}
        T_{\max} &= mg + \frac{mv^2}{L} \\
        v &= \sqrt{L\left(\frac{T_{\max}}{m} - g\right)} \approx \SI{4.60}{m/s}.
    \end{align*}
\end{enumerate}

\bigskip
%%%%%%%%%%%% CHAPTER 6 PROBLEM 6 %%%%%%%%%%%%%%%%
\paragraph*{\bf Problem 6 -- Hawk in Circular Flight}
\smallskip

\paragraph{\bf Centripetal and Tangential Accelerations}
\smallskip

\begin{enumerate}[label=(\alph*)]
    \item Given $r = \SI{11.4}{m}$, $v = \SI{3.65}{m/s}$:
    \[
        a_c = \frac{v^2}{r} \approx \SI{1.17}{m/s^2}.
    \]

    \item With tangential acceleration $a_t = \SI{1.15}{m/s^2}$:
    \begin{align*}
        a &= \sqrt{a_c^2 + a_t^2} \approx \SI{1.64}{m/s^2}, \\
        \theta &= \tan^{-1}\!\left(\frac{a_c}{a_t}\right) \approx \ang{45},
    \end{align*}
    where $\theta$ is measured from the direction of velocity \textbf{towards the center} of the circle.
\end{enumerate}

\bigskip
%%%%%%%%%%%% CHAPTER 6 PROBLEM 7 %%%%%%%%%%%%%%%%
\paragraph*{\bf Problem 7 -- Scale Reading in an Elevator}
\smallskip

\paragraph{\bf Using Two Scale Readings to Find $mg$ and $a$}
\smallskip

\begin{enumerate}[label=(\alph*)]
    \item Let $N_{\text{start}} = \SI{600}{N}$ (starting upward), $N_{\text{stop}} = \SI{394}{N}$ (stopping):
    \begin{align*}
        N_{\text{start}} &= mg + ma, \\
        N_{\text{stop}} &= mg - ma.
    \end{align*}
    Add:
    \[
        N_{\text{start}} + N_{\text{stop}} = 2mg
        \quad\Rightarrow\quad
        mg = \SI{497}{N}.
    \]
    So the person's weight is $W = \SI{497}{N}$.

    \item Mass:
    \[
        m = \frac{W}{g} = \frac{497}{9.8} \approx \SI{50.7}{kg}.
    \]

    \item Subtract the equations to find $a$:
    \begin{align*}
        N_{\text{start}} - N_{\text{stop}} &= 2ma \\
        600 - 394 &= 2ma \\
        a &= \frac{206}{2m} \approx \SI{2.03}{m/s^2}.
    \end{align*}
\end{enumerate}

%%%%%%%%%%%%%%%%% CHAPTER 13 QUIZ SECTION %%%%%%%%%%%%%%%%%%

\newpage
\paragraph{\large \bf Chapter 13 Quiz Problems } 
\bigskip
\hrule
\bigskip

%%%%%%%%%%%% CHAPTER 13 PROBLEM 1 %%%%%%%%%%%%%%%%
\paragraph*{\bf Problem 1 -- Gravitational Attraction Between Two Ships}
\smallskip

\paragraph{\bf Using Newton's Law of Gravitation to Find Acceleration}
\smallskip

Two ships of equal mass $m = 4.0\times 10^7~\si{kg}$ are separated by $r = \SI{105}{m}$.

\begin{enumerate}[label=(\alph*)]
    \item Gravitational force:
    \[
        F = G\frac{m^2}{r^2}.
    \]

    \item Acceleration of either ship:
    \[
        a = \frac{F}{m} = G\frac{m}{r^2} \approx 2.45\times 10^{-7}~\si{m/s^2}.
    \]
\end{enumerate}

\bigskip
%%%%%%%%%%%% CHAPTER 13 PROBLEM 2 %%%%%%%%%%%%%%%%
\paragraph*{\bf Problem 2 -- Mass of Jupiter from Europa's Orbit}
\smallskip

\paragraph{\bf Using Circular Orbit Data to Find Planet Mass}
\smallskip

Europa:
\[
r = 6.70\times 10^5~\si{km} = 6.70\times 10^8~\si{m}, \qquad
T = 3.55~\text{days} = 3.55\times 86400~\si{s}.
\]

For a circular orbit,
\[
\frac{GM_J}{r^2} = \frac{v^2}{r}, \qquad v = \frac{2\pi r}{T}.
\]

\begin{enumerate}[label=(\alph*)]
    \item Combine to solve for Jupiter's mass:
    \[
        M_J = \frac{4\pi^2 r^3}{G T^2} \approx 1.90\times 10^{27}~\si{kg}.
    \]
\end{enumerate}

\bigskip
%%%%%%%%%%%% CHAPTER 13 PROBLEM 3 %%%%%%%%%%%%%%%%
\paragraph*{\bf Problem 3 -- Work Done on a Meteor by Moon's Gravity}
\smallskip

\paragraph{\bf Change in Gravitational Potential from Infinity to Surface}
\smallskip

Meteor of mass $m = \SI{1075}{kg}$ falls from rest at infinity to the Moon's surface.

Moon:
\[
M_M = 7.35\times 10^{22}~\si{kg}, \quad R_M = 1.70\times 10^6~\si{m}.
\]

\begin{enumerate}[label=(\alph*)]
    \item Gravitational potential energy:
    \[
        U = -\frac{GM_M m}{r}.
    \]
    From $r = \infty$ to $r = R_M$:
    \[
        \Delta U = U_{\text{final}} - U_{\text{initial}}
        = -\frac{GM_M m}{R_M} - 0.
    \]

    \item Work done by gravity (on the meteor):
    \[
        W = -\Delta U = \frac{GM_M m}{R_M} \approx 3.10\times 10^{9}~\si{J}.
    \]
\end{enumerate}

\bigskip
%%%%%%%%%%%% CHAPTER 13 PROBLEM 4 %%%%%%%%%%%%%%%%
\paragraph*{\bf Problem 4 -- Energy to Raise a Mass to Altitude $2R_E$}
\smallskip

\paragraph{\bf Gravitational Potential Energy in the $-GMm/r$ Form}
\smallskip

Move $m = \SI{900}{kg}$ from Earth's surface ($r_1 = R_E$) to altitude $2R_E$, so $r_2 = 3R_E$.

\begin{enumerate}[label=(\alph*)]
    \item Gravitational potential:
    \[
        U = -\frac{GM_E m}{r}.
    \]
    Energy required:
    \begin{align*}
        \Delta E &= U_2 - U_1 \\
                 &= -\frac{GM_E m}{3R_E} -\left(-\frac{GM_E m}{R_E}\right) \\
                 &= GM_E m\left(\frac{1}{R_E} - \frac{1}{3R_E}\right)
                  = GM_E m\left(\frac{2}{3R_E}\right) \\
                 &\approx 3.75\times 10^{10}~\si{J}.
    \end{align*}
\end{enumerate}

\bigskip
%%%%%%%%%%%% CHAPTER 13 PROBLEM 5 %%%%%%%%%%%%%%%%
\paragraph*{\bf Problem 5 -- Orbital Period Where $g_{\text{orbit}} = 4.42~\si{m/s^2}$}
\smallskip

\paragraph{\bf Using $g_{\text{orbit}}$ as the Centripetal Acceleration}
\smallskip

At the satellite's orbit,
\[
a_c = g_{\text{orbit}} = \frac{v^2}{r}, \qquad v = \frac{2\pi r}{T}.
\]

Combine to get:
\[
g_{\text{orbit}} = \frac{4\pi^2 r}{T^2}
\quad\Rightarrow\quad
T = 2\pi\sqrt{\frac{r}{g_{\text{orbit}}}} \approx 153~\text{min}.
\]

\bigskip
%%%%%%%%%%%% CHAPTER 13 PROBLEM 6 %%%%%%%%%%%%%%%%
\paragraph*{\bf Problem 6 -- Raising a Satellite from 97 km to 209 km}
\smallskip

\paragraph{\bf Total, Kinetic, and Potential Energy Changes in Orbit}
\smallskip

Satellite mass $m = \SI{985}{kg}$.

Earth radii:
\[
r_1 = R_E + \SI{97e3}{m}, \qquad
r_2 = R_E + \SI{209e3}{m}.
\]

Total orbital energy:
\[
E = -\frac{GM_E m}{2r}.
\]

\begin{enumerate}[label=(\alph*)]
    \item \textbf{Total energy change}
    \[
        \Delta E = E_2 - E_1
        = -\frac{GM_E m}{2}\left(\frac{1}{r_2} - \frac{1}{r_1}\right)
        \approx 5.16\times 10^{8}~\si{J}
        \approx 5.16\times 10^{2}~\si{MJ}.
    \]

    \item \textbf{Kinetic energy change}
    \[
        K = +\frac{GM_E m}{2r} \quad\Rightarrow\quad
        \Delta K = K_2 - K_1 \approx -5.16\times 10^{8}~\si{J}
        \approx -5.16\times 10^{2}~\si{MJ}.
    \]

    \item \textbf{Potential energy change}
    \[
        U = -\frac{GM_E m}{r}, \qquad
        \Delta U = U_2 - U_1 \approx 1.03\times 10^{9}~\si{J}
        \approx 1.03\times 10^{3}~\si{MJ}.
    \]
\end{enumerate}

\bigskip
%%%%%%%%%%%% CHAPTER 13 PROBLEM 7 %%%%%%%%%%%%%%%%
\paragraph*{\bf Problem 7 -- Satellite at Altitude $2.08\times 10^6$ m}
\smallskip

\paragraph{\bf Circular Orbit: Period, Speed, and Acceleration}
\smallskip

Altitude:
\[
h = 2.08\times 10^6~\si{m}, \qquad
r = R_E + h.
\]

\begin{enumerate}[label=(\alph*)]
    \item \textbf{Period}
    \[
        T = 2\pi\sqrt{\frac{r^3}{GM_E}} \approx 2.15~\text{h}.
    \]

    \item \textbf{Speed}
    \[
        v = \sqrt{\frac{GM_E}{r}} \approx 6.86~\si{km/s}.
    \]

    \item \textbf{Acceleration}
    \[
        a = \frac{v^2}{r} = \frac{GM_E}{r^2} \approx 5.58~\si{m/s^2}
    \]
    directed toward the center of the Earth.
\end{enumerate}
\newpage
%========================
% Chapter 8 – Work, Energy, and Power (Key Quiz Problems)
%========================
\bigskip
\paragraph{\large \bf Chapter 8 – Key Quiz Problems}

\subsection*{P1: Block--Spring System with/without Friction (SerPSE10 8.3.OP.008)}

Given:
\[
m = 2.00~\text{kg}, \quad k = 485~\text{N/m}, \quad x_i = 5.15~\text{cm} = 0.0515~\text{m}
\]

\paragraph{(a) Frictionless surface}
Energy conservation:
\[
\frac{1}{2} k x_i^2 = \frac{1}{2} m v^2
\quad\Rightarrow\quad
v = x_i \sqrt{\frac{k}{m}}
\]
Numeric result:
\[
v \approx 0.802~\text{m/s}
\]

\paragraph{(b) With kinetic friction $\mu_k = 0.350$}
Friction force:
\[
f_k = \mu_k N = \mu_k m g
\]
Work done by friction from $x_i$ to equilibrium:
\[
W_f = f_k x_i = \mu_k m g x_i
\]
Energy balance:
\[
\frac{1}{2} k x_i^2 - W_f = \frac{1}{2} m v^2
\]
\[
\Rightarrow\quad
v = \sqrt{\frac{k x_i^2 - 2\mu_k m g x_i}{m}}
\]
Numeric result:
\[
v \approx 0.538~\text{m/s}
\]

\subsection*{P2: Power of a Car Engine (SerPSE10 8.5.P.019)}

Given:
\[
m = 1000~\text{kg}, \quad v_f = 50.0~\text{mi/h}, \quad \Delta t = 16.0~\text{s}
\]
Convert speed to SI:
\[
50.0~\text{mi/h} = 50.0 \times \frac{1609~\text{m}}{3600~\text{s}}
\approx 22.35~\text{m/s}
\]

Change in kinetic energy (starting from rest):
\[
\Delta K = \frac{1}{2} m v_f^2
\]

Average power:
\[
P = \frac{\Delta K}{\Delta t}
\]

Convert to horsepower (hp):
\[
P_{\text{hp}} = \frac{P}{746}
\]

Numeric results:
\[
\Delta K \approx 2.50\times 10^5~\text{J}
\]
\[
P \approx 1.56\times 10^4~\text{W}
\]
\[
P_{\text{hp}} \approx 20.9~\text{hp}
\]


%%%%%%%%%%%%%%%%% CONSTANTS \& KEY EQUATIONS %%%%%%%%%%%%%%%%%%

\newpage
\paragraph{\large \bf Key Constants and Equations (with Uses)} 
\bigskip
\hrule
\bigskip

\paragraph*{\bf Physical Constants (approximate)}
\smallskip
\begin{align*}
    g &\approx \SI{9.80}{m/s^2} & \text{(surface gravity on Earth)} \\
    G &\approx 6.67\times 10^{-11}~\si{N\,m^2/kg^2} & \text{(universal gravitational constant)} \\
    R_E &\approx 6.37\times 10^6~\si{m} & \text{(radius of Earth)} \\
    M_E &\approx 5.97\times 10^{24}~\si{kg} & \text{(mass of Earth)} \\
    R_M &\approx 1.70\times 10^6~\si{m} & \text{(radius of Moon)} \\
    M_M &\approx 7.35\times 10^{22}~\si{kg} & \text{(mass of Moon)} \\
\end{align*}

\bigskip
\paragraph*{\bf Newton's Laws and Basic Dynamics}
\smallskip
\begin{align*}
    \vec{F}_{\text{net}} &= m\vec{a} 
    & \text{(Newton's 2nd law; relate net force and acceleration)} \\
    W &= mg 
    & \text{(weight near Earth's surface)} \\
    N &= mg \pm ma 
    & \text{(apparent weight / scale reading in an elevator)} \\
    T &= mg + \frac{mv^2}{r} 
    & \text{(tension at bottom of a vertical circular arc)} \\
\end{align*}

\bigskip
\paragraph*{\bf Circular Motion}
\smallskip
\begin{align*}
    a_c &= \frac{v^2}{r} = \omega^2 r 
    & \text{(centripetal acceleration)} \\
    v &= \omega r 
    & \text{(relation between speed and angular speed)} \\
    \omega &= 2\pi f 
    & \text{(angular speed from frequency)} \\
    T &= \frac{1}{f} 
    & \text{(period \& frequency relation)} \\
\end{align*}

\bigskip
\paragraph*{\bf Work \& Energy}
\smallskip
\begin{align*}
    W &= \vec{F}\cdot\Delta\vec{r} 
    & \text{(work by a constant force)} \\
    K &= \frac{1}{2}mv^2 
    & \text{(kinetic energy)} \\
    \Delta K &= W_{\text{net}} 
    & \text{(work--energy theorem)} \\
    U_{\text{near Earth}} &= mgh 
    & \text{(approx.\ gravitational potential energy for small $h$)} \\
\end{align*}

\bigskip
\paragraph*{\bf Gravitation and Orbits}
\smallskip
\begin{align*}
    F_g &= G\frac{m_1 m_2}{r^2} 
    & \text{(Newton's law of universal gravitation)} \\
    U(r) &= -\dfrac{GMm}{r} 
    & \text{(gravitational potential energy for point mass / sphere)} \\
    a_g(r) &= \frac{GM}{r^2} 
    & \text{(gravitational acceleration at distance $r$)} \\
    v_{\text{orbit}} &= \sqrt{\frac{GM}{r}} 
    & \text{(speed in a circular orbit)} \\
    T_{\text{orbit}} &= 2\pi\sqrt{\frac{r^3}{GM}} 
    & \text{(period of a circular orbit)} \\
    E_{\text{orbit}} &= -\frac{GMm}{2r} 
    & \text{(total mechanical energy of a circular orbit)} \\
\end{align*}

\end{document}
