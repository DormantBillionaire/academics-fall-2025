\documentclass[12pt]{amsart}

\addtolength{\hoffset}{-2.25cm}
\addtolength{\textwidth}{4.5cm}
\addtolength{\voffset}{-2.5cm}
\addtolength{\textheight}{5cm}
\setlength{\parskip}{0pt}
\setlength{\parindent}{0in}

\usepackage{amsthm,amsmath,amssymb}
\usepackage[colorlinks = true, linkcolor = blue, citecolor = red, final]{hyperref}
\usepackage{graphicx, multicol}
\usepackage{marvosym, wasysym}
\usepackage[figurename=Figure]{caption}
\usepackage{}
\usepackage{}
\usepackage{subfig}
\usepackage{physics}
\usepackage{enumitem}
\usepackage{siunitx}

\usepackage{tikz,bm}


\newcommand{\para}{\paragraph}
\newcommand{\ds}{\displaystyle}

\pagestyle{empty}
\begin{document}
\thispagestyle{empty}
{\scshape PHY2048} \hfill {\scshape \large Cumulative Exam Review} \hfill {Fall 2025\scshape}
 
\smallskip
\hrule

\bigskip
The following Exam Review is for my Physics with General Calculus I course for the Fall of 2025, which was taken during my gap year at Palm Beach State College under Professor Leo Bae. 

\bigskip

The content covered in the final exam will be a culmination of quiz problems from the following chapters: 
\begin{itemize}
    \item Unit 1: Chapters 1, 2, 3, and 4
    \item Unit 2: Chapters 5, 6, and 13
    \item Unit 3: Chapters 7, 8, and 9
\end{itemize}
\bigskip

\hrule
\bigskip
%%%%%%%%%%%%%%%%% CHAPTER 1 QUIZ SECTION %%%%%%%%%%%%%%%%%%

\text{\large \bf Chapter 1 Quiz Problems } 
\bigskip

\paragraph*{\bf Problem 1 - Milky Way Galaxy Distance}

\paragraph{The disk of the Milky Way galaxy is about $1.0 \cross 10^{5}$ light-years (ly) in diameter. The distance from the center of the Milky Way to the center of the Andromeda galaxy is about 2.0 million ly.}
\medskip

\begin{enumerate}[label=(\alph*)]
\item Imagine a scale model where the two galaxies are represented by circular plates. If the plate representing the Milky Way has a diameter of 16 cm, what would be the distance between the centers of the two plates in meters? 
\newline

\begin{align*}
    \frac{1.0 \times 10^{5} \,ly}{2.0 \times 10^{6} \,ly} = \frac{\,diameter \,in \,cm}{x}
\end{align*}

From here, solve the proportion for x and take cm $\rightarrow$ m, where $ 1\,cm = \frac{1m}{100 cm}$
\bigskip

\item What if? The Milky Way and Andromeda galaxies are members of the Local Group, a cluster of more than 50 galaxies spread across a spherical volume with a diameter of 10 million light years. Imagine you could create a scale model of the Local Group with the Milky Way and represented as circular plates with diameters of 16 cm. What would be the diameter (in m) of your spherical scale model of the Local Group?


\begin{align*}
    \frac{1.0 \times 10^{5} \,ly}{1.0 \times 10^{7} \,ly} = \frac{\,diameter \,in \,cm}{x}
\end{align*}

From here, solve the proportion for x and take from cm $\rightarrow$ m, where $ 1\,cm = \frac{1m}{100 cm}$

\end{enumerate}

\bigskip
\hrule
\bigskip 

\paragraph*{\bf Problem 4 - Rod's and Cross Section's}
\smallskip

\paragraph{A thin rod extends from $ x = 0 $ to $x = 16.0 cm $. It has a cross-sectional area A = 8.50 $cm^{2}$,and its density increases uniformly in the positive x-direction from 3.00 $\frac{g}{cm^{3}}$ at one endpoint to 19.0 $\frac{g}{cm^{3}}$ at the other.}
\medskip

\paragraph{Note the $\rho$ at the start and the end of x (i.e $\rho$ at x = 0 and $\rho$ at x = xx cm )}
\bigskip

\begin{enumerate}[label=(\alph*)]
    \item Solve for B by plugging in C = 0 and solve the $\rho$ function
    

\begin{align*}
    \rho(0) = B + C(0) \rightarrow B = 3.00 cm^{3}
\end{align*}
    
    \item To find the C, you need to plug in x = 16cm into the $\rho$ function and solve

\begin{align*}
    \rho(x\,=\,16\,cm) = B + C(x\,=\,16\,cm)
\end{align*}

\begin{align*}
    19.0\,\frac{g}{cm^3}\, = 3.00 \frac{g}{cm^{3}} + 16C
\end{align*}

\begin{align*}
    C = 1.00\frac{g}{cm^{4}}
\end{align*}

    \item To find the \textbf{total mass}, this requires us to integrate over the whole $\rho$ function

\begin{align*}
    \int_{0}^{16}(B\,+\,Cx)(8.50cm^{2}) \cdot dx
\end{align*}

\begin{align*}
    (8.50cm^{2}) \int_{0}^{16}(Bx\,+\,\frac{Cx^{2}}{2})
\end{align*}

\begin{align*}
    (8.50cm^{2}) \int_{0}^{16}((3.00)x\,+\,\frac{(1.00)x^{2}}{2})
\end{align*}

\begin{align*}
    (8.50cm^{2}) \int_{0}^{16}((3.00)(16.00)\,+\,\frac{(1.00)(16.00)^{2}}{2})
\end{align*}

\begin{align*}
    (8.50cm^{2}) \int_{0}^{16}(48\,+128)
\end{align*}

\begin{align*}
    (8.50cm^{2})(176)\rightarrow\,1496\,g
\end{align*}

\begin{align*}
    1496 g \frac{1 kg}{1000 g} \, = 1.496\,kg
\end{align*}

\end{enumerate}

%%%%%%%%%%%%%%%%% CHAPTER 2 QUIZ SECTION %%%%%%%%%%%%%%%%%%

\hrule
\bigskip
\text{\large \bf Chapter 2 Quiz Problems } 
\paragraph*{\bf Problem 2 - Constant Speed}

\begin{enumerate}[label=(\alph*)]
    \item Given \textbf{Cartesian}

\begin{align*}
    x = rcos(\theta) \,\,\,\, y = rsin(\theta)
\end{align*}
    
    \item Given \textbf{Polar}

\begin{align*}
    \rho(x\,=\,16\,cm) = B + C(x\,=\,16\,cm)
\end{align*}

\begin{align*}
    19.0\,\frac{g}{cm^3}\, = 3.00 \frac{g}{cm^{3}} + 16C
\end{align*}

\begin{align*}
    C = 1.00\frac{g}{cm^{4}}
\end{align*}

    \item To find the \textbf{total mass}, this requires us to integrate over the whole $\rho$ function

\begin{align*}
    \int_{0}^{16}(B\,+\,Cx)(8.50cm^{2}) \cdot dx
\end{align*}

\begin{align*}
    (8.50cm^{2}) \int_{0}^{16}(Bx\,+\,\frac{Cx^{2}}{2})
\end{align*}

\begin{align*}
    (8.50cm^{2}) \int_{0}^{16}((3.00)x\,+\,\frac{(1.00)x^{2}}{2})
\end{align*}

\begin{align*}
    (8.50cm^{2}) \int_{0}^{16}((3.00)(16.00)\,+\,\frac{(1.00)(16.00)^{2}}{2})
\end{align*}

\begin{align*}
    (8.50cm^{2}) \int_{0}^{16}(48\,+128)
\end{align*}

\begin{align*}
    (8.50cm^{2})(176)\rightarrow\,1496\,g
\end{align*}

\begin{align*}
    1496 g \frac{1 kg}{1000 g} \, = 1.496\,kg
\end{align*}

\end{enumerate}

\paragraph*{\bf Problem 15 - Hellicopters \& Heights}
\bigskip
\begin{enumerate}[label=(\alph*)]
    \item Given \textbf{Cartesian}

\begin{align*}
    x = rcos(\theta) \,\,\,\, y = rsin(\theta)
\end{align*}
    
    \item Given \textbf{Polar}

\begin{align*}
    \rho(x\,=\,16\,cm) = B + C(x\,=\,16\,cm)
\end{align*}

\begin{align*}
    19.0\,\frac{g}{cm^3}\, = 3.00 \frac{g}{cm^{3}} + 16C
\end{align*}

\begin{align*}
    C = 1.00\frac{g}{cm^{4}}
\end{align*}

    \item To find the \textbf{total mass}, this requires us to integrate over the whole $\rho$ function

\begin{align*}
    \int_{0}^{16}(B\,+\,Cx)(8.50cm^{2}) \cdot dx
\end{align*}

\begin{align*}
    (8.50cm^{2}) \int_{0}^{16}(Bx\,+\,\frac{Cx^{2}}{2})
\end{align*}

\begin{align*}
    (8.50cm^{2}) \int_{0}^{16}((3.00)x\,+\,\frac{(1.00)x^{2}}{2})
\end{align*}

\begin{align*}
    (8.50cm^{2}) \int_{0}^{16}((3.00)(16.00)\,+\,\frac{(1.00)(16.00)^{2}}{2})
\end{align*}

\begin{align*}
    (8.50cm^{2}) \int_{0}^{16}(48\,+128)
\end{align*}

\begin{align*}
    (8.50cm^{2})(176)\rightarrow\,1496\,g
\end{align*}

\begin{align*}
    1496 g \frac{1 kg}{1000 g} \, = 1.496\,kg
\end{align*}

\end{enumerate}

%%%%%%%%%%%%%%%%% CHAPTER 3 QUIZ SECTION %%%%%%%%%%%%%%%%%%

\bigskip
\paragraph{\large \bf Chapter 3 Quiz Problems } 
\bigskip
\hrule
\bigskip
\paragraph*{\bf Problem 1, 2, \& 3 - Polar $\rightarrow$ Cartesian \& Back}
\smallskip

\paragraph{converting to cartesian to polar}
\smallskip

\begin{enumerate}[label=(\alph*)]
    \item Given \textbf{Polar}, where $r =$ xx $cm$ and cos = $\theta$

\begin{align*}
    x = rcos(\theta) \,\,\,\, y = rsin(\theta)
\end{align*}
    
    \item Given \textbf{Partial Cartesian}, where $ x $ or $y$ $=$ xx in $m$ and $ y = $ xx in degrees $\theta$

\begin{align*}
    x = rcos(\theta) \rightarrow (3.00m) = r\cdot \cos(\ang{45})
\end{align*}

    \item Solve for r and then use r and the angle to solve for y

\begin{align*}
    r = \frac{3.00m}{\cos (\ang{45})}
\end{align*}

\begin{align*}
    y = r\cdot\sin(\ang{45}) \,\,\,OR \,\,\, y = \frac{x}{\tan(\ang{45})}
\end{align*}

\end{enumerate}


\bigskip
\paragraph*{\bf Problem 4 - Forces and Boxes}
\bigskip

\paragraph{\bf Vector Addition of Two Forces On a Box}
\smallskip

\bigskip

\begin{figure}[h!]
    \centering
    \includegraphics[width=0.35\linewidth]{/Users/aspenjohnson/Documents/GitHub/academics-fall-2025/PHY2048_GeneralPhysicsI/figures/Chapter3_ForcesAndBoxes.png}
    \caption{Forces $\vec{F_1}$ and $\vec{F_2}$ acting on a box at an angle $\theta$}
    \label{fig:TwoStringTwoForces}
\end{figure}

\begin{enumerate}[label=(\alph*)]
    \item Given $\vec{F_1}$ and $\vec{F_2}$ both have magnitudes of $\vec{F_1}$ and $\vec{F_2}$ as well as an angle $\theta$, we can\textbf{ resolve each vector into it's components}

\begin{align*}
    \vec{F_x}=F\cdot \cos(\theta)\,\,\,and\,\,\, \vec{F_y}=F\cdot \sin(\theta) 
\end{align*}
    
\begin{align*}
    \vec{F}_{1x}=F_{1}\cdot \cos(\theta_{1})\,\,\, and \,\,\, \vec{F}_{1y}=F_{1}\cdot \sin(\theta_{1})
\end{align*}
  

\begin{align*}
    \vec{F}_{2x}=0 \,\,\, and \,\,\, \vec{F}_{2y}=F_{2}
\end{align*}

    \item Therefore we can calculate the \textbf{resultant vector, magnitude}, as well as \textbf{angle}
\begin{align*}
    F_{Rx}=F_{1x}+F_{2x}
\end{align*}

\begin{align*}
    F_{Ry}=F_{1y}+F_{2y}
\end{align*}


\begin{align*}
    F_{R}=\sqrt{F^{2}_Rx+F^{2}_Ry}
\end{align*}

\begin{align*}
    \theta_R = \tan^-1(\frac{F_Ry}{F_Rx})
\end{align*}

%%%%%%%%%%%%% COME BCK TO WRITE OUT A SIMPLE EXAMPLE %%%%%%%%%%%%%

\begin{align*}
    y = r\cdot\sin(\ang{45}) \,\,\,OR \,\,\, y = \frac{x}{\tan(\ang{45})}
\end{align*}

\end{enumerate}
%%%%%%%%%%%%%%%%% CHAPTER 5 QUIZ SECTION %%%%%%%%%%%%%%%%%%

\bigskip
\paragraph{\large \bf Chapter 5 Quiz Problems } 
\bigskip
\hrule
\bigskip
\paragraph*{\bf Problem 1 - Resultant Forces \& More Magnitudes }
\smallskip

\paragraph{converting to cartesian to polar}
\smallskip

\begin{enumerate}[label=(\alph*)]
    \item Given \textbf{Polar}, where $r =$ xx $cm$ and cos = $\theta$

\begin{align*}
    x = rcos(\theta) \,\,\,\, y = rsin(\theta)
\end{align*}
    
    \item Given \textbf{Partial Cartesian}, where $ x $ or $y$ $=$ xx in $m$ and $ y = $ xx in degrees $\theta$

\begin{align*}
    x = rcos(\theta) \rightarrow (3.00m) = r\cdot \cos(\ang{45})
\end{align*}

    \item Solve for r and then use r and the angle to solve for y

\begin{align*}
    r = \frac{3.00m}{\cos (\ang{45})}
\end{align*}

\begin{align*}
    y = r\cdot\sin(\ang{45}) \,\,\,OR \,\,\, y = \frac{x}{\tan(\ang{45})}
\end{align*}

\end{enumerate}


\bigskip
\paragraph*{\bf Problem 2 - Normal Forces By Ground On Seemingly Normal Boxes }
\smallskip

\paragraph{\bf Vector Addition of Two Forces On a Box}
\smallskip

\bigskip



\begin{enumerate}[label=(\alph*)]
    \item Given $\vec{F_1}$ and $\vec{F_2}$ both have magnitudes of $\vec{F_1}$ and $\vec{F_2}$ as well as an angle $\theta$, we can\textbf{ resolve each vector into it's components}

\begin{align*}
    \vec{F_x}=F\cdot \cos(\theta)\,\,\,and\,\,\, \vec{F_y}=F\cdot \sin(\theta) 
\end{align*}
    
\begin{align*}
    \vec{F}_{1x}=F_{1}\cdot \cos(\theta_{1})\,\,\, and \,\,\, \vec{F}_{1y}=F_{1}\cdot \sin(\theta_{1})
\end{align*}
  

\begin{align*}
    \vec{F}_{2x}=0 \,\,\, and \,\,\, \vec{F}_{2y}=F_{2}
\end{align*}

    \item Therefore we can calculate the \textbf{resultant vector, magnitude}, as well as \textbf{angle}
\begin{align*}
    F_{Rx}=F_{1x}+F_{2x}
\end{align*}

\begin{align*}
    F_{Ry}=F_{1y}+F_{2y}
\end{align*}
\end{enumerate}

\begin{align*}
    F_{R}=\sqrt{F^{2}_Rx+F^{2}_Ry}
\end{align*}

\begin{align*}
    \theta_R = \tan^-1(\frac{F_Ry}{F_Rx})
\end{align*}

\bigskip
\paragraph*{\bf Problem 3 - Sailboats \& Constant Velocities}




\paragraph{\bf Vector Addition of Two Forces On a Box}
\smallskip

\begin{figure}[h!]
    \centering
    \includegraphics[width=0.35\linewidth]{/Users/aspenjohnson/Documents/GitHub/academics-fall-2025/PHY2048_GeneralPhysicsI/figures/Chapter5_SailboatsAndForces.png}
    \caption{Sailboat being pulled by Forces $\vec{F}, \vec{n},\,and \,\vec{P}$}
    \label{fig:SailboatForces}
\end{figure}

\bigskip



\begin{enumerate}[label=(\alph*)]
    \item Given $\vec{F_1}$ and $\vec{F_2}$ both have magnitudes of $\vec{F_1}$ and $\vec{F_2}$ as well as an angle $\theta$, we can\textbf{ resolve each vector into it's components}

\begin{align*}
    \vec{F_x}=F\cdot \cos(\theta)\,\,\,and\,\,\, \vec{F_y}=F\cdot \sin(\theta) 
\end{align*}
    
\begin{align*}
    \vec{F}_{1x}=F_{1}\cdot \cos(\theta_{1})\,\,\, and \,\,\, \vec{F}_{1y}=F_{1}\cdot \sin(\theta_{1})
\end{align*}
  

\begin{align*}
    \vec{F}_{2x}=0 \,\,\, and \,\,\, \vec{F}_{2y}=F_{2}
\end{align*}

    \item Therefore we can calculate the \textbf{resultant vector, magnitude}, as well as \textbf{angle}
\begin{align*}
    F_{Rx}=F_{1x}+F_{2x}
\end{align*}

\begin{align*}
    F_{Ry}=F_{1y}+F_{2y}
\end{align*}
\end{enumerate}

\begin{align*}
    F_{R}=\sqrt{F^{2}_Rx+F^{2}_Ry}
\end{align*}

\begin{align*}
    \theta_R = \tan^-1(\frac{F_Ry}{F_Rx})
\end{align*}

\bigskip
\paragraph*{\bf Problem 4 - Stabilizing Broken Legs}



\paragraph{\bf Vector Addition of Two Forces On a Box}
\smallskip

\begin{figure}[h!]
    \centering
    \includegraphics[width=0.35\linewidth]{/Users/aspenjohnson/Documents/GitHub/academics-fall-2025/PHY2048_GeneralPhysicsI/figures/Chapter5_StabilizingABrokenLeg.png}
    \caption{Sailboat being pulled by Forces $\vec{F}, \vec{n},\,and \,\vec{P}$}
    \label{fig:myimage}
\end{figure}

\bigskip



\begin{enumerate}[label=(\alph*)]
    \item Given $\vec{F_1}$ and $\vec{F_2}$ both have magnitudes of $\vec{F_1}$ and $\vec{F_2}$ as well as an angle $\theta$, we can\textbf{ resolve each vector into it's components}

\begin{align*}
    \vec{F_x}=F\cdot \cos(\theta)\,\,\,and\,\,\, \vec{F_y}=F\cdot \sin(\theta) 
\end{align*}
    
\begin{align*}
    \vec{F}_{1x}=F_{1}\cdot \cos(\theta_{1})\,\,\, and \,\,\, \vec{F}_{1y}=F_{1}\cdot \sin(\theta_{1})
\end{align*}
  

\begin{align*}
    \vec{F}_{2x}=0 \,\,\, and \,\,\, \vec{F}_{2y}=F_{2}
\end{align*}

    \item Therefore we can calculate the \textbf{resultant vector, magnitude}, as well as \textbf{angle}
\begin{align*}
    F_{Rx}=F_{1x}+F_{2x}
\end{align*}

\begin{align*}
    F_{Ry}=F_{1y}+F_{2y}
\end{align*}
\end{enumerate}

\begin{align*}
    F_{R}=\sqrt{F^{2}_Rx+F^{2}_Ry}
\end{align*}

\begin{align*}
    \theta_R = \tan^-1(\frac{F_Ry}{F_Rx})
\end{align*}

\newpage
\paragraph*{\bf Problem 5 - Speed Over Time Curves}
\bigskip


\paragraph{\bf Vector Addition of Two Forces On a Box}
\medskip

\begin{figure}[h!]
    \centering
    \includegraphics[width=0.35\linewidth]{/Users/aspenjohnson/Documents/GitHub/academics-fall-2025/PHY2048_GeneralPhysicsI/figures/Chapter5_SpeedOverTimeCurve.png}
    \caption{Speed Over Time Curve}
    \label{fig:SpeedOverTime}
\end{figure}

\bigskip



\begin{enumerate}[label=(\alph*)]
    \item Given $\vec{F_1}$ and $\vec{F_2}$ both have magnitudes of $\vec{F_1}$ and $\vec{F_2}$ as well as an angle $\theta$, we can\textbf{ resolve each vector into it's components}

\begin{align*}
    \vec{F_x}=F\cdot \cos(\theta)\,\,\,and\,\,\, \vec{F_y}=F\cdot \sin(\theta) 
\end{align*}
    
\begin{align*}
    \vec{F}_{1x}=F_{1}\cdot \cos(\theta_{1})\,\,\, and \,\,\, \vec{F}_{1y}=F_{1}\cdot \sin(\theta_{1})
\end{align*}
  

\begin{align*}
    \vec{F}_{2x}=0 \,\,\, and \,\,\, \vec{F}_{2y}=F_{2}
\end{align*}

    \item Therefore we can calculate the \textbf{resultant vector, magnitude}, as well as \textbf{angle}
\begin{align*}
    F_{Rx}=F_{1x}+F_{2x}
\end{align*}

\begin{align*}
    F_{Ry}=F_{1y}+F_{2y}
\end{align*}
\end{enumerate}

\begin{align*}
    F_{R}=\sqrt{F^{2}_Rx+F^{2}_Ry}
\end{align*}

\begin{align*}
    \theta_R = \tan^-1(\frac{F_Ry}{F_Rx})
\end{align*}


\paragraph*{\bf Problem 6 - Elevators Experiencing Tension}
\bigskip


\paragraph{\bf Vector Addition of Two Forces On a Box}
\medskip

\begin{figure}[h!]
    \centering
    \includegraphics[width=0.30\linewidth]{/Users/aspenjohnson/Documents/GitHub/academics-fall-2025/PHY2048_GeneralPhysicsI/figures/Chapter5_ElevatorsAndTension.png}
    \caption{Two masses, $m$ suspended in an elevator, uphel by two $T_1$ and $T_2$}
    \label{fig:elevatortension}
\end{figure}

\bigskip



\begin{enumerate}[label=(\alph*)]
    \item Given $\vec{F_1}$ and $\vec{F_2}$ both have magnitudes of $\vec{F_1}$ and $\vec{F_2}$ as well as an angle $\theta$, we can\textbf{ resolve each vector into it's components}

\begin{align*}
    \vec{F_x}=F\cdot \cos(\theta)\,\,\,and\,\,\, \vec{F_y}=F\cdot \sin(\theta) 
\end{align*}
    
\begin{align*}
    \vec{F}_{1x}=F_{1}\cdot \cos(\theta_{1})\,\,\, and \,\,\, \vec{F}_{1y}=F_{1}\cdot \sin(\theta_{1})
\end{align*}
  

\begin{align*}
    \vec{F}_{2x}=0 \,\,\, and \,\,\, \vec{F}_{2y}=F_{2}
\end{align*}

    \item Therefore we can calculate the \textbf{resultant vector, magnitude}, as well as \textbf{angle}
\begin{align*}
    F_{Rx}=F_{1x}+F_{2x}
\end{align*}

\begin{align*}
    F_{Ry}=F_{1y}+F_{2y}
\end{align*}
\end{enumerate}

\begin{align*}
    F_{R}=\sqrt{F^{2}_Rx+F^{2}_Ry}
\end{align*}

\begin{align*}
    \theta_R = \tan^-1(\frac{F_Ry}{F_Rx})
\end{align*}
%%%%%%%%%%%%%%%%% CHAPTER 6 QUIZ SECTION %%%%%%%%%%%%%%%%%%


\bigskip
\paragraph{\large \bf Chapter 6 Quiz Problems } 
\bigskip
\hrule
\bigskip
%%%%%%%%%%%% CHAPTER 6 QUESTION 1 %%%%%%%%%%%%%%%%
\paragraph*{\bf Problem 1 -Station Rotation}
\smallskip

\paragraph{converting to cartesian to polar}
\smallskip

\begin{enumerate}[label=(\alph*)]
    \item Given \textbf{Polar}, where $r =$ xx $cm$ and cos = $\theta$

\begin{align*}
    x = rcos(\theta) \,\,\,\, y = rsin(\theta)
\end{align*}
    
    \item Given \textbf{Partial Cartesian}, where $ x $ or $y$ $=$ xx in $m$ and $ y = $ xx in degrees $\theta$

\begin{align*}
    x = rcos(\theta) \rightarrow (3.00m) = r\cdot \cos(\ang{45})
\end{align*}

    \item Solve for r and then use r and the angle to solve for y

\begin{align*}
    r = \frac{3.00m}{\cos (\ang{45})}
\end{align*}

\begin{align*}
    y = r\cdot\sin(\ang{45}) \,\,\,OR \,\,\, y = \frac{x}{\tan(\ang{45})}
\end{align*}

\end{enumerate}

%%%%%%%%%%%% CHAPTER 6 QUESTION 2 %%%%%%%%%%%%%%%%

\bigskip
\paragraph*{\bf Problem 2 - Coins on a turntable}
\bigskip

\paragraph{\bf Vector Addition of Two Forces On a Box}
\smallskip

\bigskip



%%%%%%%%%%%%%%%%% CHAPTER 13 QUIZ SECTION %%%%%%%%%%%%%%%%%%


\bigskip
\paragraph{\large \bf Chapter 13 Quiz Problems } 
\bigskip
\hrule
\bigskip
\paragraph*{\bf Problem 1, 2, \& 3 - Polar $\rightarrow$ Cartesian \& Back}
\smallskip

\paragraph{converting to cartesian to polar}
\smallskip

\begin{enumerate}[label=(\alph*)]
    \item Given \textbf{Polar}, where $r =$ xx $cm$ and cos = $\theta$

\begin{align*}
    x = rcos(\theta) \,\,\,\, y = rsin(\theta)
\end{align*}
    
    \item Given \textbf{Partial Cartesian}, where $ x $ or $y$ $=$ xx in $m$ and $ y = $ xx in degrees $\theta$

\begin{align*}
    x = rcos(\theta) \rightarrow (3.00m) = r\cdot \cos(\ang{45})
\end{align*}

    \item Solve for r and then use r and the angle to solve for y

\begin{align*}
    r = \frac{3.00m}{\cos (\ang{45})}
\end{align*}

\begin{align*}
    y = r\cdot\sin(\ang{45}) \,\,\,OR \,\,\, y = \frac{x}{\tan(\ang{45})}
\end{align*}

\end{enumerate}


\bigskip
\paragraph*{\bf Problem 4 - Forces and Boxes}
\bigskip

\paragraph{\bf Vector Addition of Two Forces On a Box}
\smallskip

\bigskip


\begin{enumerate}[label=(\alph*)]
    \item Given $\vec{F_1}$ and $\vec{F_2}$ both have magnitudes of $\vec{F_1}$ and $\vec{F_2}$ as well as an angle $\theta$, we can\textbf{ resolve each vector into it's components}

\begin{align*}
    \vec{F_x}=F\cdot \cos(\theta)\,\,\,and\,\,\, \vec{F_y}=F\cdot \sin(\theta) 
\end{align*}
    
\begin{align*}
    \vec{F}_{1x}=F_{1}\cdot \cos(\theta_{1})\,\,\, and \,\,\, \vec{F}_{1y}=F_{1}\cdot \sin(\theta_{1})
\end{align*}
  

\begin{align*}
    \vec{F}_{2x}=0 \,\,\, and \,\,\, \vec{F}_{2y}=F_{2}
\end{align*}

    \item Therefore we can calculate the \textbf{resultant vector, magnitude}, as well as \textbf{angle}
\begin{align*}
    F_{Rx}=F_{1x}+F_{2x}
\end{align*}

\begin{align*}
    F_{Ry}=F_{1y}+F_{2y}
\end{align*}


\begin{align*}
    F_{R}=\sqrt{F^{2}_Rx+F^{2}_Ry}
\end{align*}

\begin{align*}
    \theta_R = \tan^-1(\frac{F_Ry}{F_Rx})
\end{align*}

\end{enumerate}





\end{document}
