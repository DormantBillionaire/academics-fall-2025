%%%%%%%%%%%%%%%%%%%%%%%%%%%%%%%%%%%%
%      PHYSICS 2048L LAB 3 REPORT 
%      ASPEN JOHNSON
%      FALL 2025 ~ DR. LEO BAE LAB  
%%%%%%%%%%%%%%%%%%%%%%%%%%%%%%%%%%%%

%%%%%%%%%%%%%%%%%%%%%%%%%%%%%%%%%%%%
%          Notes on This Document
%
%. This is beyond very ambitious for me as someone who has not done ANYTHING with LaTeX
%,but I think that this assignment will allow me the space to learn LaTeX and the 
% momentum to continue to execute notes and assignments
%
%%%%%%%%%%%%%%%%%%%%%%%%%%%%%%%%%%%%

\documentclass[11pt,letterpaper]{report}

% --------- Packages --------
\usepackage{newtxtext, newtxmath}
\usepackage[margin=1in]{geometry}       % 1 inch margins on all sides
\usepackage{setspace}                   % For line spacing 
\usepackage{microtype}                  % Better spacing and justification
\usepackage{titlesec}                   % For Section Title Formatting 
\usepackage{graphicx}                   % For figures
\usepackage{caption}                    % Better captions 
\usepackage{hyperref}                   % Clickable references and links 
\usepackage{fancyhdr}                   % Custom headers/ footers
\usepackage{bm}                         % Bold mathematical symbols 
\usepackage{ragged2e}                   %For the Ragged Right package title alignment


% --------- Formatting --------

\setstretch{1.15}
\setlength{\parskip}{6pt}
\setlength{\parindent}{0pt}

% --------- Section Style & Heading (Unnumbered + entered)  --------
\titleformat{\section}
    {\large\bfseries\raggedright}{}{0em}{}

\titleformat{\subsection}
    {\normalsize\bfseries\itshape\raggedright}{}{0em}{}

% ---------- (Macros) Formatting Commands ---------
\newcommand{\para}{\paragraph{}}

% --------- Header & Footer --------
\pagestyle{fancy}
\fancyhf{}
\fancyhead[C]{ PHY2048 Exam 01 Cumulative Notes}
\fancyfoot[C]{\thepage}

% --------- Title Block --------
\title{{\Large PHY2048 Examination 01 } \\
{\large Cumulative Lecture}}

\author{Aspen J. Johnson\\
Palm Beach State Community College}
% \title{\huge\vspace{-1cm}\textbf{PHY2048 Lab 3 Report}\\[10pt]
% \large Determining Spring Constants (k) using Hooke's Law \\[2pt]
%Aspen J. Johnson\\[4pt]
% Palm Beach State Community College\vspace{-0.5cm}}
\date{}


% --------- Document Begins --------

\begin{document}   
\maketitle
\vspace{1cm}
\newpage                                % Creates new page --> Will push the below to the next page



% --------- Section 1: Introduction --------
\section{1. Introduction}
\para
The objective of this experiment was to determine the spring constant $(k)$ for a single spring system using Hooke’s Law. The spring system used most closely resembled an extension spring and was measured prior to adding varying masses. The displacements were measured using a wooden ruler. After the commencement of the experiment, the force $(F)$ was plotted against the extension $(x)$, and the slope of the resulting linear graph provided the spring constant for the single spring with various masses added.

\para
In this experiment, we take on the role of an engineering design team in which a toy company must ensure that the maximum force supplied by a spring mechanism inside a toy gun does not exceed $20\,\mathrm{N}$ for safety reasons. The spring constant $(k)$ obtained, if accurate, represents how the spring will behave when placed under increased load, thereby ensuring safe and reliable use in other mechanical systems.




\end{document}