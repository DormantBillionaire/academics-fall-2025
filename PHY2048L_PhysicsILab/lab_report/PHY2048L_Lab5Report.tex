%%%%%%%%%%%%%%%%%%%%%%%%%%%%%%%%%%%%
%      PHYSICS 2048L LAB 3 REPORT 
%      ASPEN JOHNSON
%      FALL 2025 ~ DR. LEO BAE LAB  
%%%%%%%%%%%%%%%%%%%%%%%%%%%%%%%%%%%%

%%%%%%%%%%%%%%%%%%%%%%%%%%%%%%%%%%%%
%          Notes on This Document
%
%. This is beyond very ambitious for me as someone who has not done ANYTHING with LaTeX
%,but I think that this assignment will allow me the space to learn LaTeX and the 
% momentum to continue to execute notes and assignments
%
%%%%%%%%%%%%%%%%%%%%%%%%%%%%%%%%%%%%

\documentclass[11pt,letterpaper]{article}

% --------- Packages --------
\usepackage{newtxtext, newtxmath}
\usepackage[margin=1in]{geometry}       % 1 inch margins on all sides
\usepackage{setspace}                   % For line spacing 
\usepackage{microtype}                  % Better spacing and justification
\usepackage{titlesec}                   % For Section Title Formatting 
\usepackage{graphicx}                   % For figures
\usepackage{caption}                    % Better captions 
\usepackage{hyperref}                   % Clickable references and links 
\usepackage{fancyhdr}                   % Custom headers/ footers
\usepackage{bm}                         % Bold mathematical symbols 
\usepackage{ragged2e}                   %For the Ragged Right package title alignment


% --------- Formatting --------

\setstretch{1.15}
\setlength{\parskip}{6pt}
\setlength{\parindent}{0pt}

% --------- Section Style & Heading (Unnumbered + entered)  --------
\titleformat{\section}
    {\large\bfseries\raggedright}{}{0em}{}

\titleformat{\subsection}
    {\normalsize\bfseries\itshape\raggedright}{}{0em}{}

% ---------- (Macros) Formatting Commands ---------
\newcommand{\para}{\paragraph{}}

% --------- Header & Footer --------
\pagestyle{fancy}
\setlength{\headheight}{14pt} % Fix for fancyhdr warning
\fancyhf{}
\fancyhead[C]{PHY2048L Lab 5 Latent Heat Fusion}
\fancyfoot[C]{\thepage}

% --------- Title Block --------
\title{{\Large PHY2048L Lab 5 Heat Fusion} \\
{\large Latent Heat Of Fusion}}

\author{Aspen J. Johnson\\
Palm Beach State Community College}
% \title{\huge\vspace{-1cm}\textbf{PHY2048 Lab 3 Report}\\[10pt]
% \large Determining Spring Constants (k) using Hooke's Law \\[2pt]
%Aspen J. Johnson\\[4pt]
% Palm Beach State Community College\vspace{-0.5cm}}
\date{}


% --------- Document Begins --------

\begin{document}   
\maketitle
\vspace{1cm}
\newpage                                % Creates new page --> Will push the below to the next page



% --------- Section 1: Introduction --------
\section{1. Introduction}

\para
The purpose of this experiment was to determine two key thermal properties of a phase-changing material (PCM): 1. its melting temperature, and 2. its latent heat of fusion. These quantities were obtained by the PHY2048L lab experiment heating a PCM headband in a controlled water bath while monitoring temperature changes over 20 second increments. During the melting period, the PCM maintains a nearly constant temperature, allowing the melting point to be measured directly. 

\para
After establishing the melting temperature, in alignment with part 1 of the lab, the experiment proceeded to quantify the thermal energy absorbed by the PCM during melting by analyzing the water’s temperature rise under a known heater power. This known heating power was on level 3 of the burner equipment we used to heat the pot that possessed the water necessary to carry out this experiment. This allowed for determining the total heat input $Q$, the heat absorbed by the water, and finally the latent heat of fusion using:
\[
L_f = \frac{Q - m_{w} c_{w} \Delta T_{w}}{m_{\mathrm{PCM}}}.
\]

\para 
It should also be noted that the other section of Dr. Leo Bae's PHY2048L course, provided a plausible estimate for Power in Watts, which allowed for the calculations below.
\para
Thermal energy, internal energy, work, and power all play central roles in this lab. Thermal energy refers to the internal kinetic and potential energy of molecules, work describes energy transfer, and power quantifies the rate of energy delivery. The concepts of specific heat, heat of fusion, and phase changes were reviewed using the Part 1 lecture slides prior to the experiments execution.

\para
On Monday, November 24, 2025, although the lab section did re-measured the PCM headband under improved conditions, the measurements were only partially used as the time was the main fruits of our labor. The heating power of $270\,\mathrm{W}$ was used to calculate total heat input from $Q = P \Delta t$. 2 thermometers were used to record water temperature every fixed time interval while the PCM melted, and mass measurements for both the PCM and the surrounding water were taken for later calculations. This report presents the background, experimental setup, calculations, and conclusions for determining the latent heat of fusion of the PCM.


% --------- Section 2: Principles --------
\section{2. Principles}

\para
Thermal energy is the internal energy associated with the random motion and interactions of atoms and molecules. This internal energy consists of kinetic contributions (translational, rotational, vibrational motion) and potential contributions arising from intermolecular forces, expressed as:
\[
E_{\text{internal}} = E_{K} + E_{P}.
\]
When a substance absorbs heat, its internal energy increases, which may make itself plain as a temperature increase or a phase transition ocurrs.

\para
\textbf{Heat, Work, and Power.}
Heat $Q$ is the transfer of thermal energy between systems due to temperature differences. Work corresponds to energy transfer resulting from forces acting over distances. Power is the rate of energy transfer, defined as:
\[
P = \frac{Q}{\Delta t}.
\]
In this experiment, knowing the heater power ($270\,\mathrm{W}$) allowed for computing the total thermal energy delivered during melting.

\para
\textbf{Specific Heat.}
The specific heat capacity $c$ of a substance quantifies the energy required to raise the temperature of $1\,\mathrm{kg}$ by $1^\circ\mathrm{C}$. Heat transferred to water in this experiment is described by:
\[
Q_{w} = m_{w} c_{w} \Delta T_{w}.
\]

\para
\textbf{Latent Heat and Phase Change.}
During phase transitions, temperature remains constant even as heat continues to be absorbed. This energy does not raise temperature but instead alters the microscopic structure, such as breaking intermolecular bonds during melting. The latent heat of fusion $L_f$ is the energy required to melt $1\,\mathrm{kg}$ of material:
\[
Q_{\mathrm{melt}} = m_{\mathrm{PCM}} L_f.
\]
This lab focuses on determining $L_f$ for the PCM headband by analyzing how much energy remains after accounting for the water's temperature rise.

\para
The theoretical concepts described above were directly relevant to analyzing the data collected during Part 1 of the lab assignment and performing the calculations completed on November 24, 2025.



% --------- Section 3: Experimental Setup --------
\section{3. Experimental Setup}

\para
The experiment involved heating a PCM headband immersed in a pot of water placed on an electric stovetop set to level 3. A digital scale was used to measure the mass of the PCM and the water. A thermometer (initially two or three thermometers) recorded temperature readings at fixed intervals while the PCM melted.

\para
Early attempts overheated the PCM too quickly due to excessive heat transfer, resulting in seemingly unusable temperature data, which was later on used to compile the "Temperature Over Time" graph later on in this paper. The procedure was adjusted by reducing the number of thermometers, lengthening the measurement interval, and stabilizing the heating rate. These improvements were implemented in the repeated trial conducted on Monday, November 24, 2025.

\para
The final procedure consisted of:
\begin{itemize}
    \item Measuring the mass of the PCM: \textbf{193.9 g = 0.1939 kg}.
    \item Measuring the mass of the surrounding water: \textbf{489.9 g = 0.4899 kg}.
    \item Recording the water’s initial and final temperatures.
    \item Applying a heater with known power: \textbf{$270\,\mathrm{W}$}.
    \item Recording temperature every fixed time interval while the PCM melted at a constant temperature plateau.
    \item Measuring the heating duration: \textbf{587.2 seconds}.
\end{itemize}

\para
This setup ensures sufficient heat transfer to melt the PCM while keeping the melting interval long enough to observe the temperature plateau and accurately measure the thermal energy delivered.


% --------- Section 4: Results --------
\section{4. Results}

\para
The recorded measurements and calculated results are summarized below. All computational steps were performed using the Excel sheet created during lab.

\begin{itemize}
    \item Mass of PCM: $0.1939\,\mathrm{kg}$
    \item Mass of water: $0.4899\,\mathrm{kg}$
    \item Temperature change of water: $\Delta T_{w} = 50.9^\circ\mathrm{C}$
    \item Specific heat of water: $c_{w} = 4.19 \times 10^{3}\,\mathrm{J/(kg\,^{\circ}C)}$
    \item Heater power: $P = 270\,\mathrm{W}$
    \item Heating time: $\Delta t = 587.2\,\mathrm{s}$
\end{itemize}

\para
The total thermal energy delivered by the heater was:
\[
Q = P \Delta t = 270 \times 587.2 = 1.58544 \times 10^{5}\,\mathrm{J}.
\]

\para
The thermal energy absorbed by the water was:
\[
Q_{w} = m_{w} c_{w} \Delta T_{w}
      = 0.4899 \times (4.19 \times 10^{3}) \times 50.9
      = 1.04481 \times 10^{5}\,\mathrm{J}.
\]

\para
The remaining energy was absorbed by the PCM during melting:
\[
Q_{\mathrm{melt}} = Q - Q_{w} = 5.4063 \times 10^{4}\,\mathrm{J}.
\]

\para
Finally, the latent heat of fusion for the PCM was calculated as:
\[
L_{f} = \frac{Q_{\mathrm{melt}}}{m_{\mathrm{PCM}}}
      = \frac{5.4063 \times 10^{4}}{0.1939}
      \approx 2.79 \times 10^{5} \,\mathrm{J/kg}.
\]


% --------- Section 5: Discussion --------
\section{5. Discussion}


\para
The melting temperature letup observed during data collection confirmed that the PCM undergoes a phase transition at a nearly constant temperature, consistent with the historical behavior of phase-changing materials. The calculated latent heat of fusion, approximately $2.8 \times 10^{5}\,\mathrm{J/kg}$, is comparable in magnitude to that of water ($3.33 \times 10^{5}\,\mathrm{J/kg}$), suggesting the PCM behaves similarly to water-based materials often used in thermal-type regulation devices.

\para
One source of experimental error includes heat loss to the surrounding air, which reduces the effective thermal energy reaching the PCM and water. Additionally, thermometer lag and inconsistent stirring may have influenced temperature uniformity. Variations in stovetop output and heat conduction through the container also contributed to uncertainty. Despite these limitations, the overall results were consistent and physically reasonable.

\para
The first trial produced unreliable data due to excessively rapid heating, underscoring the importance of controlled thermal input. Adjustments in the second trial resolved this issue and allowed for more accurate observation of the phase-change plateau and resulting calculations.
\begin{figure}[h!]
    \centering
    \includegraphics[width=0.9\linewidth]{/Users/aspenjohnson/Documents/GitHub/academics-fall-2025/PHY2048L_PhysicsILab/figures/TempOverTimeGraph.png}
    \caption{Temperature of PCM over time for the initial trial.}
    \label{fig:failed_trial}
\end{figure}


% --------- Section 6: Conclusion --------
\section{6. Conclusion}

\para
This experiment successfully determined the melting temperature and latent heat of fusion of a PCM headband. By measuring the mass of the PCM, recording the temperature change of the surrounding water, and calculating total heat input from the heater power and heating duration, the latent heat of fusion was found to be approximately:
\[
L_{f} \approx 2.8 \times 10^{5}\,\mathrm{J/kg}.
\]

\para
The results are consistent with expected values for commercially used PCM materials. The experiment provided practical experience with heat transfer, energy conservation, and phase-change physics, while also demonstrating the importance of controlled heating conditions and precise temperature measurements.

\para
\textit{Disclaimer: Portions of this report were refined with the assistance of AI-based proofreading tools and the LaTeX online PDF guide. The analysis, calculations, and final interpretations remain entirely my own work.}

\para
Figure~\ref{fig:failed_trial} illustrates the failed temperature retrieval attempt made by PHY2048L on the first round of tests

\end{document}