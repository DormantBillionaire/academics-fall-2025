%%%%%%%%%%%%%%%%%%%%%%%%%%%%%%%%%%%%
%      PHYSICS 2048L LAB 3 REPORT 
%      ASPEN JOHNSON
%      FALL 2025 ~ DR. LEO BAE LAB  
%%%%%%%%%%%%%%%%%%%%%%%%%%%%%%%%%%%%

%%%%%%%%%%%%%%%%%%%%%%%%%%%%%%%%%%%%
%          Notes on This Document
%
%. This is beyond very ambitious for me as someone who has not done ANYTHING with LaTeX
%,but I think that this assignment will allow me the space to learn LaTeX and the 
% momentum to continue to execute notes and assignments
%
%%%%%%%%%%%%%%%%%%%%%%%%%%%%%%%%%%%%

\documentclass[11pt,letterpaper]{article}

% --------- Packages --------
\usepackage{newtxtext, newtxmath}
\usepackage[margin=1in]{geometry}       % 1 inch margins on all sides
\usepackage{setspace}                   % For line spacing 
\usepackage{microtype}                  % Better spacing and justification
\usepackage{titlesec}                   % For Section Title Formatting 
\usepackage{graphicx}                   % For figures
\usepackage{caption}                    % Better captions 
\usepackage{hyperref}                   % Clickable references and links 
\usepackage{fancyhdr}                   % Custom headers/ footers
\usepackage{bm}                         % Bold mathematical symbols 
\usepackage{ragged2e}                   %For the Ragged Right package title alignment


% --------- Formatting --------

\setstretch{1.15}
\setlength{\parskip}{6pt}
\setlength{\parindent}{0pt}

% --------- Section Style & Heading (Unnumbered + entered)  --------
\titleformat{\section}
    {\large\bfseries\raggedright}{}{0em}{}

\titleformat{\subsection}
    {\normalsize\bfseries\itshape\raggedright}{}{0em}{}

% ---------- (Macros) Formatting Commands ---------
\newcommand{\para}{\paragraph{}}

% --------- Header & Footer --------
\pagestyle{fancy}
\fancyhf{}
\fancyhead[C]{ PHY2048L Lab 5 Latent Heat Fusion}
\fancyfoot[C]{\thepage}

% --------- Title Block --------
\title{{\Large PHY2048L Lab 5 Heat Fusion} \\
{\large Latent Heat Of Fusion}}

\author{Aspen J. Johnson\\
Palm Beach State Community College}
% \title{\huge\vspace{-1cm}\textbf{PHY2048 Lab 3 Report}\\[10pt]
% \large Determining Spring Constants (k) using Hooke's Law \\[2pt]
%Aspen J. Johnson\\[4pt]
% Palm Beach State Community College\vspace{-0.5cm}}
\date{}


% --------- Document Begins --------

\begin{document}   
\maketitle
\vspace{1cm}
\newpage                                % Creates new page --> Will push the below to the next page



% --------- Section 1: Introduction --------
\section{1. Introduction}
\para Below I will discuss topics that need to be introduced and defined in this section of the lab report. 
\begin{itemize}
    \item Thermal Energy, which is the \textbf{"internal energy"} associated with the rapid movement of atoms and molecules. 
    \item \textbf{Internal Energy} increases as an object or substance is gradually heated up, and the random motions of all atoms or molecules inside become more energetic to increase the internal energy.
    \item Internal energy is stored as two types: Potential and Kinetic Energy as given by:  $E_{Internal} = E_{Kinetic} + E_{Potential}$
    \item Speak about and define \textbf{work, energy, and power} as well as their units 
    \item Talk about and define Status Change Of Materials 
    \item Define and discuss \textbf{Specific Heat, Latenet Heat Of Fusion, and Heat of Vaporization}
    \item Read through all of Part 1 from the PPT lecture, Part 2 will not be done for this term/ semester.
    \item On Monday, November 24th 2025 we did not redo the lab, instead we measured and calculated q, and C for the specific heat 
    \item We measured the PCM Headband again and got accurate results,relative to last week
    \item We utilized one thermometer instead of two this time, and recorded the temperature of the thermometer after a certain number of seconds denpting the temperature for the PCM headband to melt at a specific time 
    \item We then calculated Q, heat, by multiplying time in seconds by the Power in Watts at 270 provided by the other lab section under Dr. Bae 
    \item It is also possible to do a lab uncertainty analysis 
    \item 
\end{itemize}
\para
The objective of this experiement is to plot the graph of the phase changes as it pertains to heat and rates of change 
as the experimental group tracked the increase in temperature overtime of the PCM headband, in a pot of water. 

\para
In this experiment, (we did not take on the role of any engineering team, as far as I know, will confirm later)


% --------- Section 2: Principles --------
\section{2. Principles}


% --------- Section 3: Experimental Setup --------
\section{1. Experimental Setup}
\para Below I will detail the experiemnt, as best as I can remember
\begin{itemize}
    \item Once in Lab, we were given a PPT lectures so that we could have a foundation of what to do for the Lab, what materials were going to be used, and how the process would transpire. 
    \item We used a type of stove top burner and placed it on level 3 for the first trial of the experiment. 
    \item We initially ran into issues with the water heating the PCM too quickly and therefore messing with the readings and data of our experiment, this caused us to have to re-do the experiment as well as plan to do another full trial on Monday, November 24th 2025.
    \item Before placing the PCM into the water, we measured the temperature of the water, but I did not record this temperature in my excel \textbf{(Need to get this from a lab member)}.
    \item Initially we had three thermometers, and had 1 in the water to keep equilibrium while stiring the water, while the other 2 are inside of the PCM during the process.
    \item The three thermometers were reduced to two for the first completed (but unsuccessful as far as data) trial, to allow for ease of readings as temperatures climbed rapidly.
    \item During the unsuccessful initial trial, we went ahead and measured every 10 seconds, but this proved to be too quick of a pace with threee temp readings. 
    \item During the successful first trial, we used only two thermometers and measured every 20 seconds and took the readings to the nearest 0.1 . 
    \item We measured both trials with the stovetop burner on a level of 3. 
    \item 
\end{itemize}
% --------- Section 4: Results --------
\section{4. Results}

% --------- Section 5: Discussion --------
\section{5. Discussion}
\para This is the section where the failed lab results will go, including the graph 

% --------- Section 6: Conclusion --------
\section{6. Conclusion}
\end{document}