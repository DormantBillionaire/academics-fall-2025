%%%%%%%%%%%%%%%%%%%%%%%%%%%%%%%%%%%%%%%%%%%%%%
%    LINEAR ALGEBRA – NETWORK FLOW PROJECT
%    ASPEN JOHNSON
%    FALL 2025 ~ PBSC
%%%%%%%%%%%%%%%%%%%%%%%%%%%%%%%%%%%%%%%%%%%%%%

\documentclass[11pt,letterpaper]{article}

% ---------- Packages ----------
\usepackage{newtxtext, newtxmath}
\usepackage[margin=1in]{geometry}
\usepackage{setspace}
\usepackage{microtype}
\usepackage{titlesec}
\usepackage{graphicx}
\usepackage{caption}
\usepackage{hyperref}
\usepackage{fancyhdr}
\usepackage{bm}
\usepackage{ragged2e}
\usepackage{amsmath}

% ---------- Formatting ----------
\setstretch{1.2}
\setlength{\parskip}{6pt}
\setlength{\parindent}{0pt}

\titleformat{\section}
  {\large\bfseries\raggedright}{}{0em}{}

\titleformat{\subsection}
  {\normalsize\bfseries\itshape\raggedright}{}{0em}{}

% ---------- Header/Footer ----------
\pagestyle{fancy}
\setlength{\headheight}{14pt} % Fix for fancyhdr warning
\fancyhf{}
\fancyhead[C]{Linear Algebra – Network Flow Project}
\fancyfoot[C]{\thepage}

% ---------- Title ----------
\title{
{\Large MAS2103-3 – Network Flow Project}\\[4pt]
{\large Traffic Flow System with Nodes}
}
\author{Aspen J. Johnson\\Palm Beach State College}
\date{}

% ---------- Document Begins ----------
\begin{document}
\maketitle
\newpage

%%%%%%%%%%%%%%%%%%%%%%%%%%%%%%%%%%%%%%%%%%%%%%%%%%%%%%%%%%%%%
\section{1. Introduction}

This project explores a Network Flow system using concepts from Linear Algebra, specifically the use of linear systems, coefficient matrices, and Gaussian elimination. My main objective is to design a realistic traffic network consisting of five nodes, write its corresponding system of equations based on flow conservation, form the coefficient matrix, and solve the system completely.

Traffic flow networks provide a practical example of linear systems in which every intersection (node) must satisfy the condition that the total inflow equals the total outflow. By modeling one-way roads as directed edges and vehicle rates as variables, the system can be represented algebraically and solved for all possible patterns of traffic movement through the network.

This report constructs a five-node downtown traffic model, derives the associated flow equations, and demonstrates a complete step-by-step solution using Gaussian elimination.

%%%%%%%%%%%%%%%%%%%%%%%%%%%%%%%%%%%%%%%%%%%%%%%%%%%%%%%%%%%%%
\section{2. Network Description}
\begin{figure}[h!]
    \centering
    \includegraphics[width=0.85\linewidth]{/Users/aspenjohnson/Documents/GitHub/academics-fall-2025/MAS2103_LinearAlgebra/figures/Node_ProjectImage.jpeg}
    \caption{My hand-drawn diagram of the five-node traffic flow network showing the incoming 100 cars at Node A, the directed flow variables \(x_1, x_2, x_3, x_4, x_5\), and the exit rates at Nodes B, C, D, and E.}
    \label{fig:network_diagram}
\end{figure}

We model traffic flow among five intersections labeled A, B, C, D, and E. A total of 100 cars per minute enter the system at Node A. At various nodes, cars leave the network:


\begin{itemize}
    \item 40 cars/min exit at Node B
    \item 30 cars/min exit at Node C
    \item 20 cars/min exit at Node D
    \item 10 cars/min exit at Node E
\end{itemize}

The remaining cars continue traveling along one-way streets in the network. The unknown traffic flows are:

\[
x_1 : A \to B,\quad
x_2 : A \to C,\quad
x_3 : B \to D,\quad
x_4 : C \to D,\quad
x_5 : D \to E.
\]



%%%%%%%%%%%%%%%%%%%%%%%%%%%%%%%%%%%%%%%%%%%%%%%%%%%%%%%%%%%%%
\section{3. System of Equations}

Using the conservation rule \textbf{inflow = outflow} at each node, we obtain:

\subsection*{Node A}
\[
x_1 + x_2 = 100
\]

\subsection*{Node B}
\[
x_1 = 40 + x_3
\]

\subsection*{Node C}
\[
x_2 = 30 + x_4
\]

\subsection*{Node D}
\[
x_3 + x_4 = 20 + x_5
\]

\subsection*{Node E}
\[
x_5 = 10
\]

This produces a system of five linear equations in five unknowns.

%%%%%%%%%%%%%%%%%%%%%%%%%%%%%%%%%%%%%%%%%%%%%%%%%%%%%%%%%%%%%
\section{4. Coefficient Matrix}

Writing each equation in standard form:

\[
\begin{aligned}
x_1 + x_2 &= 100\\
x_1 - x_3 &= 40\\
x_2 - x_4 &= 30\\
x_3 + x_4 - x_5 &= 20\\
x_5 &= 10
\end{aligned}
\]

The coefficient matrix $A$, variable vector $\mathbf{x}$, and constant vector $\mathbf{b}$ are:

\[
A=
\begin{bmatrix}
1 & 1 & 0 & 0 & 0\\
1 & 0 & -1 & 0 & 0\\
0 & 1 & 0 & -1 & 0\\
0 & 0 & 1 & 1 & -1\\
0 & 0 & 0 & 0 & 1
\end{bmatrix},
\qquad
\mathbf{x}=
\begin{bmatrix}
x_1\\ x_2\\ x_3\\ x_4\\ x_5
\end{bmatrix},
\qquad
\mathbf{b}=
\begin{bmatrix}
100\\ 40\\ 30\\ 20\\ 10
\end{bmatrix}.
\]

Thus the matrix equation is:
\[
A\mathbf{x}=\mathbf{b}.
\]

%%%%%%%%%%%%%%%%%%%%%%%%%%%%%%%%%%%%%%%%%%%%%%%%%%%%%%%%%%%%%
\section{5. Solving the System (Gaussian Elimination)}

We form the augmented matrix:

\[
\left[
\begin{array}{ccccc|c}
1 & 1 & 0 & 0 & 0 & 100\\
1 & 0 & -1 & 0 & 0 & 40\\
0 & 1 & 0 & -1 & 0 & 30\\
0 & 0 & 1 & 1 & -1 & 20\\
0 & 0 & 0 & 0 & 1 & 10
\end{array}
\right].
\]

\subsection*{Row Operations}

(1) \(R_2 \leftarrow R_2 - R_1\):
\[
R_2 = [0,-1,-1,0,0,-60].
\]

(2) Multiply \(R_2\) by \(-1\):
\[
R_2 = [0,1,1,0,0,60].
\]

(3) \(R_3 \leftarrow R_3 - R_2\):
\[
R_3 = [0,0,-1,-1,0,-30].
\]

(4) Multiply \(R_3\) by \(-1\):
\[
R_3 = [0,0,1,1,0,30].
\]

(5) Eliminate the 1 in column 3 of \(R_4\):
\[
R_4 \leftarrow R_4 - R_3 = [0,0,0,0,-1,-10].
\]

(6) Add \(R_5\) to \(R_4\):
\[
R_4 = [0,0,0,0,0,0].
\]

(7) Eliminate the 1 in column 2 of \(R_1\):
\[
R_1 \leftarrow R_1 - R_2 = [1,0,0,1,0,70].
\]

The final row-reduced form is:

\[
\left[
\begin{array}{ccccc|c}
1 & 0 & 0 & 1 & 0 & 70\\
0 & 1 & 0 & -1 & 0 & 30\\
0 & 0 & 1 & 1 & 0 & 30\\
0 & 0 & 0 & 0 & 0 & 0\\
0 & 0 & 0 & 0 & 1 & 10
\end{array}
\right].
\]

%%%%%%%%%%%%%%%%%%%%%%%%%%%%%%%%%%%%%%%%%%%%%%%%%%%%%%%%%%%%%
\section{6. General Solution}

From the reduced system, we obtain:

\[
\begin{aligned}
x_1 + x_4 &= 70,\\
x_2 - x_4 &= 30,\\
x_3 + x_4 &= 30,\\
x_5 &= 10.
\end{aligned}
\]

Let the free variable be
\[
x_4 = t.
\]

Then the full solution is:

\[
\boxed{
\begin{aligned}
x_1 &= 70 - t,\\
x_2 &= 30 + t,\\
x_3 &= 30 - t,\\
x_4 &= t,\\
x_5 &= 10
\end{aligned}}
\qquad \text{for any real } t.
\]

To make sense in the traffic context, all flows must be nonnegative:

\[
\begin{aligned}
x_1 = 70 - t &\ge 0 \Rightarrow t \le 70,\\
x_2 = 30 + t &\ge 0 \Rightarrow t \ge -30,\\
x_3 = 30 - t &\ge 0 \Rightarrow t \le 30,\\
x_4 = t &\ge 0 \Rightarrow t \ge 0.
\end{aligned}
\]

Combining these inequalities, we obtain:
\[
0 \le t \le 30.
\]

Thus every physically meaningful traffic pattern in this network is described by the solution above with \(0 \le t \le 30\).

%%%%%%%%%%%%%%%%%%%%%%%%%%%%%%%%%%%%%%%%%%%%%%%%%%%%%%%%%%%%%
\section{7. Conclusion}

This project modeled a traffic network using a system of linear equations. By applying the conservation-of-flow principle at each node, a five-equation system in five unknowns was constructed and written in matrix form as \(A\mathbf{x} = \mathbf{b}\). Gaussian elimination was then used to row-reduce the augmented matrix and obtain a one-parameter family of solutions.

The final solution shows how different choices of the free variable \(t\) correspond to different, but consistent, traffic flow patterns through the network, as long as all flows remain nonnegative. This demonstrates how Linear Algebra provides a powerful framework for analyzing real-world systems such as transportation networks.

\par
\textit{Disclaimer: This report's structure and wording were refined with the assistance of AI-based proofreading tools and general LaTeX references. All mathematical reasoning, computations, and interpretations are my own work, and were supported by the following video walkthrough: \href{https://youtu.be/eHoKS4Ccd74?si=7HaWx7pFMYk17dIx}{Network Flow Explanation}, as well as office hours with Professor Abdullah.}

\end{document}
